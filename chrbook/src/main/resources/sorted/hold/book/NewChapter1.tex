\chapter{Greetings - <s>unalihelijeti</s>}

\section{Greet Others - <s></s>}
\subsection{What You Will Learn}
In this unit you will learn:
\begin{itemize}
    \item How to greet people
    \item How to make short descriptive phrases
    \item About definites and indefinites
    \item Say "hello" and "good  bye"
    \item Ask how someone is doing
\end{itemize}
\vfill
\subsection{Vocabulary}
\input{chap1vocab1.tex}

\subsection{Hello - <s>osiyo</s>}

In Cherokee there is only one way to say 'Good Morning,' 'Good Afternoon,' 'Good Evening,' and Hello; that is by saying <s>osiyo</s> \  \textcolor{red} {(o-)si-yo} or the shortened version <s>siyo</s> \ \textcolor{red} {si-yo}.
\footnote{In this book we will follow the convention of placing optional syllables in parenthesis.  You will see this written as (O)siyo.  The parentheses around the 'O' mean that the voicing of the 'O' is optional.}
\footnote{Osi will be discussed more in the section \hyperref[sec:wordBreakdownTohiOsi]{Word Breakdown - Tohi and Osi}}

\subsection{Exercise - <s></s>}
\noindent Translate to Cherokee Syllabary and the Latin equivalent \newline
1. Hello, Mary 2. Hello, Mark 3. Hello, Daniel

\subsection{Good Bye - <s>donadagohvi</s>}
There is no word for 'goodbye' only 'to meet again'. The way to say 'goodbye' to one person is <s>donadagohvi</s> \  \textcolor{red}{do-na-da-go-hv-i}. If you would like to say 'good bye' to more than one person you would say <s>dodadagohvi</s> \ \textcolor{red}{do-da-da-go-hv-i}.  Lit: Let's meet again. \footnote{We will discuss the plurality prefixes (d-) in the section \hyperref[sec:wordBreakdownPluralityPrefixes]{"Word Breakdown - Plurality Prefixes"}}

\subsection{Exercise - <s></s>}
\noindent 1. Good bye, Mary and John 2. Good bye, Titus 3. Good bye, Daniel 4. Good
bye, Mary, John, Susan, and Mark\\

\subsection{Exercise - <s></s>}
\noindent picture of a person at a house waving good bye to people in a car \\
\noindent picture of the people in a car waving good bye to a person at the house\\
\noindent person shaking hands with someone business attire\\
\noindent person shaking hands with someone casual attire (maybe a bar)

\subsection{Exercise - <s></s>}
Fill in the missing words:\\
1. <s>jani</s> ---------\\
1. I'm (my name is) John.

\subsection{Vocabulary}
\input{chap1vocab2.tex}

\subsection{Word Breakdown - <s>dohi</s> and <s>osi</s> Tohi and Osi}
\label{sec:wordBreakdownTohiOsi}
Altman and Belt (pp91-92) have this to say about Tohi and Osi:

Tohi is a Cherokee morpheme that indicates the state in which nature is flowing at its appropriate
pace and everything is as it should be. This fundamental concept
is used in greetings and responses (“Tohi-gwa-tsv?” “Tohi-gwu.”), and in
a variety of other instances and constructions that indicate an underlying
concern with the notion that things be flowing well in the Cherokee
world. Tohi can be glossed variously as “well,” “peaceful,” “unhurried,”
and “health.” In the Cherokee speakers’ view, if the state of tohi becomes
disrupted there can be disastrous consequences, and communities that
are disrupted in this way can be dangerous or unhealthy places to live.

In addition to and as an adjunct to tohi, the concept of osi describes
the proper state of the individual person. Visualized as upright, facing
forward, and resting on a single point of balance, osi is also used in
greetings and replies (“osi-gwa-tsv?” “osi-gwu.”), and in other contexts
that indicate that the notion of an individual’s state of being is crucial
in ensuring that all is flowing well in the larger Cherokee world. Osi is
properly understood as referring to the state of neutrality and balance,
but it is most often glossed as “good.” If individuals are out of balance,
they can cause problems in the larger system.
\cite{altmanBelt9192}

\subsection{Dialect Breakdown - Otali vs Giduwah}
The Giduwah, or Eastern, dialect of Cherokee varies in some ways from the Otali, or Western, dialect dialect of Cherokee.  A simple example is <s>hawa</s> (G) vs <s>howa</s> (O).  Different spellings, same word.  Both mean "ok, alright, sure" \cite{joynerlesson4} The word “<s>howa</s>” is an affirmative response and can be understood to mean different things depending on how it is used. Two of the more common meanings are “Okay” and “You are welcome”. \cite{joynerlesson4}

\subsection{-ju ending -<s>-ju</s>}
\noindent <s>-ju</s> \textcolor{red}{-ju} “It is, isn’t it?”\\
\indent When we add “\textcolor{red}{-ju} ” to the end of the first word in a sentence we are asking “It is, isn’t it?”. This ending is used when expecting a “yes” or similarly confirming answer.\\
\indent Examples:\\
\indent <s>Sagwu</s> \  \textcolor{red} {sa-gwu} “One.”\\
\indent \indent <s>Sagwuju</s>?\ \textcolor{red}{sa-qwu-ju} “It is one, isn’t it?”\\
\indent <s>Howa</s> \textcolor{red}{ho-wa} . “Ok.”\\
\indent \indent <s>Howaju</s>? \ \textcolor{red}{ho-wa-ju?}  “It is OK, isn’t it?”\\ \\

\subsection{-gwu ending - <s>-gwu</s>}
<s>-gwu</s> \textcolor{red} {-gwu} “Just. Only.”\\
\indent When we add “<s>-gwu</s>” to the end of a word, it is like adding “just” or “only” before the word in English.\\
\indent Examples:\\
\indent <s>Sagwu</s>.  “One.”\\
\indent \indent <s>Sagwugwu.</s> /textcolor{red} {sa-gwu-gwu} “It is just one.”\\
\indent Osda. “Good.”\\
\indent \indent Osdagwu. “It is only good.”\\

\subsection{-na ending -<s>-na</s> \cite{joyner21}}
<s>-na</s> \textcolor{red} {-na} “What of? What about?”\\
\indent When we add “-na” to the end of the first word in a sentence, we indicate we are asking “What about? What of?”.\\
\indent Examples:\\
\indent <s>Sagwu.</s> “One.”\\
\indent \indent <s>Sagwuna?</s> \textcolor{red} {sa-gwu-na} “What about the one?”\\
\indent <s>Ayv.</s> \textcolor{red} {a-yv} “I/me.”\\
\indent \indent <s>Ayvna?</s> /textcolor{red}{Ayvna?} “What about me?”

\subsection{-hv ending -<s>-hv</s>}
<s>-hv</s> \textcolor{red}{-hv} “But?”\\
\indent When we add “<s>-hv</s>” to the end of the first word in a sentence, we indicate we are asking “But for?”.\\
\indent Examples:\\
\indent <s>Sagwu</s>  “One.”\\
\indent \indent <s>Sagwuhv?</s> \textcolor{red} {sa-gwu-hv?}  “But for the one?”\\
\indent <s>Nihi. </s> \textcolor{red} {ni-hi} “You.”\\
\indent \indent <s>Nihihv?</s> \textcolor{red} {ni-hi-hv?}  “But for you?”\cite{joyner21}

%\includepdf[pages=7]{../../cherokee/lessons/cherokeelessonsjoyner/volume-01-chapter-02.pdf}
When we combine these special endings with our existing vocabulary, we greatly expand what we can express.
\cite{joyner21}

\subsection{-dv ending - <s>-dv</s>}
<s>-dv</s> ending

\subsection{-ju, -s, -sgi, -ke endings <s>-ju, -s, -sgi, -ke</s>}
Question markers

\subsection{<s>tla</s>, <s>vtla</s>, <s>tlahv</s> - \textcolor{red}{tla}, \textcolor{red}{vtla}, \textcolor{red}{tlahv}}
tlahv is emphatic - may also be like no, thankyou\\
pp6 smith tla yi (or i) negation elaborate more\\
negation tla + y before vowels\\
tla y(i) before consonants and verbs\\
e.g. tla yigoliga - I don't understand \cite{SmithHolmespp32}
\\
no, not so - tla, tlahv, vtla -- when the answer is 'no' Cherokee don't usually say "tla" only they say "tlahv" or "vtla"
\cite{SmithHolmespp6}

\subsection{Exercise - <s></s>}
\noindent 1. Ok 2. I am fine 3. And you? 4. Thank you 5. You're welcome

\subsection{Listening Comprehension - <s></s>}
\noindent Listen to greetings and good bye dialog. See what words you recognize.

\subsection{Vocabulary}
\input{chap1vocab3.tex}

\subsection{Exercise - <s></s>}
\noindent 1. Good, Susan 2. Very good, John\\

\subsection{Listening Comprehension - <s></s>}
\noindent listen to dialog answer questions

\subsection{Complete Vocabulary List - <s></s>}
\begin{tabular}{p{3.5cm} p{11cm}}
    \input{chap1vocab1.tex}
    \input{chap1vocab2.tex}
    \input{chap1vocab3.tex}
\end{tabular}

\subsection{Answers for matching sections - <s></s>}
\noindent Answers for the make correct dialog exchange picture Answers for the dialog section questions of the listening comprehension