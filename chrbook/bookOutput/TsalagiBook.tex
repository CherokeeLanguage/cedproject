%&latex
\documentclass{book}
% for reworking Cherokee so I don't have to define the font every time
%https://tex.stackexchange.com/questions/132087/displaying-cherokee-text

%formatting ideas
%https://anneurai.net/2017/10/18/thesis-formatting-in-latex/

%to include pdfs in this document
\usepackage{pdfpages}

%\usepackage[a4paper, total={3in, 5in}]{geometry}
\usepackage[a4paper,margin=0.5in,landscape]{geometry}

%+MakeIndex
\usepackage{imakeidx}

\makeindex
%-MakeIndex

\usepackage{graphicx}
\usepackage{hyperref}
\usepackage{verbatim}
\usepackage{color}
\usepackage[USenglish]{babel}
\usepackage{fullpage}
\tracinglostchars=2
%\usepackage{babel}
\usepackage{unicode-math}
\usepackage{ucharclasses}
\usepackage{hyperref}
\usepackage{verbatim}
\usepackage{color}
%\usepackage{polyglossia}

%multiple columns for text
\usepackage{multicol}

%advanced footnote settings
\usepackage{footnote}

%double column footnotes
%\usepackage{dblfnote}


\babelprovide[import=chr]{cherokee}

\defaultfontfeatures{ Scale=MatchUppercase, Ligatures=TeX }

%% Aboriginal Serif, Aboriginal Sans and Digohweli, by Christopher Harvey, are
%% available at http://www.languagegeek.com/font/fontdownload.html
%% These fonts do not support the lowercase Cherokee added in Unicode 8.0.
\defaultfontfeatures[AboriginalSerif]{
    Ligatures=Common,
    UprightFont = *REGULAR ,
    BoldFont = *BOLD ,
    ItalicFont = *ITALIC ,
    BoldItalicFont = *BOLDITALIC ,
    Extension = .ttf }
\defaultfontfeatures[AboriginalSans]{
    Ligatures= {Common, Discretionary},
    UprightFont = *REGULAR ,
    BoldFont = *BOLD ,
    ItalicFont = *ITALIC ,
    BoldItalicFont = *BOLDITALIC ,
    Extension = .ttf }
%% An alternative is Noto Sans Cherokee, availabe at
%% https://www.google.com/get/noto/#sans-cher
%% This font comes in numerous weights, so select the pair that matches your
%% other fonts, e.g. Medium and ExtraBold or Light and Semibold. It has no
%% italics.
\defaultfontfeatures[NotoSansCherokee]{
    UprightFont=*-Regular ,
    BoldFont = *-Bold ,
    Extension = .ttf
}
%% Everson Mono is available at: https://www.evertype.com/emono/
\babelfont{rm}
[Ligatures=Common, Scale=1.0]{Noto Sans}
\babelfont[cherokee]{rm}
{NotoSansCherokee}
\babelfont{sf}
[Ligatures=Common]{Noto Sans}
\babelfont[cherokee]{sf}
{NotoSansCherokee}
\babelfont{tt}
{Courier New}
%\setmathfont{TeX Gyre Termes Math}


%\setDefaultTransitions{\selectlanguage{english}}{}
\setTransitionsForCherokeeFull{\selectlanguage{cherokee}}{\selectlanguage{english}}

%\font \Cherokee = "Noto Sans Cherokee"
\begin{document}

%+Title
    \title{\Huge\bf ᏣᎳᎩ ᎦᏬᏂᎯᏍᏓ ᏗᎪᏪᎵ\\Cherokee Language Book}
    \author{Tim Orr}
    \date{\today}
    \maketitle
%-Title

%+Contents
    \tableofcontents
%+Contents

%+Make Parts
%\part{Lessons}
%-Make Parts

% this removes the numbering inside the chapters leaving only the first number - if you want to display 1.2.13 then set the 0 to 2
\setcounter{secnumdepth}{0}
%\sffamily

\index{Greet Others}
\index{ᎤᎾᎵᎮᎵᏤᏘ}
\chapter{Greet Others - ᎤᎾᎵᎮᎵᏤᏘ}
\index{What You Will Learn}\subsection{What You Will Learn}
In this unit you will learn:
\begin{itemize}
\item How to greet people
\item How to make short descriptive phrases
\item About definites and indefinites
\item Say "hello" and "goodbye"
\item Ask how someone is doing
\end{itemize}\newpage

\subsection{Dialog - }
\begin{tabular}{p{2cm} p{11cm}}
ᏓᏂᎵ:\newline \textcolor{red}{Danili}: & ᎣᏏᏲ.  ᏓᏂᎵ ᏓᏩᏙᎠ.  ᎦᏙ ᏕᏣᏙᎠ? 
\newline\textcolor{red}{Osiyo.  Danili dawadoa.  Gado dejadoa?}\\
ᏑᏌᏂ:\newline \textcolor{red}{Susani}: & ᎣᏏᏲ.  ᏑᏌᏂ ᏓᏩᏙᎠ.  ᏙᎯᏧ? 
\newline\textcolor{red}{Osiyo.  Susani dawadoa.  Dohiju?}\\
ᏓᏂᎵ:\newline \textcolor{red}{Danili}: & ᎣᏏᏊ.  ᏂᎯᎾᎲ? 
\newline\textcolor{red}{Osigwu.  Nihinahv?}\\
ᏑᏌᏂ:\newline \textcolor{red}{Susani}: & ᎣᏍᏓ.  ᏙᎾᏓᎪᎲᎢ. 
\newline\textcolor{red}{Osda.  Donadagohvi.}\\
ᏓᏂᎵ:\newline \textcolor{red}{Danili}: & ᏙᎾᏓᎪᎲᎢ. 
\newline\textcolor{red}{Donadagohvi.}\\
\end{tabular}
\\
\\
\\
\noindent\begin{tabular}{p{2cm} p{11cm}}Daniel: & Hello.  My name is Daniel.  What is your name?\\
Susan: & Hello.  My name is Susan.  How are you?\\
Daniel: & I'm fine.  And you?\\
Susan: & Good.  Goodbye.\\
Daniel: & Goodbye.\\
\end{tabular}
\subsection{Vocabulary - ᏗᎧᏁᎢᏍᏗ 
}
\begin{tabular}{p{3cm} p{11cm}}
Titus & ᏓᏓᏏ 
 \newline \textcolor{red}{Dadasi}\\
Timothy & ᏗᎹᏗ 
 \newline \textcolor{red}{Dimadi}\\
Mary & ᎺᎵ 
 \newline \textcolor{red}{Meli}\\
Mark & ᎹᎦ 
 \newline \textcolor{red}{Maga}\\
Daniel & ᏓᏂᎵ 
 \newline \textcolor{red}{Danili}\\
Susan & ᏑᏌᏂ 
 \newline \textcolor{red}{Susani}\\
John & ᏣᏂ 
 \newline \textcolor{red}{Jani}\\
Hello & (Ꭳ)ᏏᏲ 
 \newline \textcolor{red}{(o)siyo}\\
And & ᎠᎴ 
  or ᏃᎴ 
 \newline \textcolor{red}{ale <e>or nole}\\
Good & ᎣᏍᏓ 
 \newline \textcolor{red}{osda}\\
Well/Fine & ᎣᏏᏊ 
 \newline \textcolor{red}{osigwu}\\
\end{tabular}

\index{Hello}\subsection{Hello - ᎣᏏᏲ}
In Cherokee there is only one way to say 'Good Morning,' 'Good Afternoon,' 'Good Evening,' and Hello; that is by saying ᎣᏏᏲ \textcolor{red}{(o)siyo} or the shortened version ᏏᏲ \textcolor{red}{siyo}.\footnote{We will follow the convention of placing optional syllables in parenthesis.  You will see this written as (O)siyo.  The parentheses around the 'O' mean that the voicing of the 'O' is optional.}\footnote{Osi will be discussed more in the section \hyperref[sec:wordBreakdownTohiOsi]{Word Breakdown - Tohi and Osi}}
\section{Exercise - ᎠᎵᏏᎾᎯᏍᏗᏍᎬ ᏗᎬᏙᏗ}
Translate to Cherokee syllabary and the phonetic equivalent\\
1. Hello, Mary 2. Hello, Mark 3. Hello, Daniel 4. Hello, Susan\\
1. (O)siyo, Meli. 2. (O)siyo, Maga. 3. (O)siyo, Danili 4. (O)siyo, Susani\\
\index{Goodbye}\subsection{Goodbye - ᏙᎾᏓᎪᎲᎢ}
There is no word for 'goodbye' only 'to meet again'. The way to say 'goodbye' to one person is ᏙᎾᏓᎪᎲᎢ \textcolor{red}{donadagohvi}. If you would like to say 'goodbye' to more than one person you would say ᏙᏓᏓᎪᎲᎢ \textcolor{red}{dodadagohvi}.  Lit: Let's meet again.\footnote{We will discuss the plurality prefixes (d-) in the section \hyperref[sec:wordBreakdownPluralityPrefixes]{Word Breakdown - Plurality Prefixes}}
\section{Exercise - ᎠᎵᏏᎾᎯᏍᏗᏍᎬ ᏗᎬᏙᏗ}
Translate to Cherokee syllabary and the phonetic equivalent\\
1. Goodbye, Mary and John. 2. Goodbye, Titus. 3. Goodbye, Daniel. 4. Goodbye, Mary, John, Susan, and Mark.\\
1. Dodadagohvi, Meli ale Jani. 2. Donadagohvi, Dadasi 3. Donadagohvi, Danili 4. Dodadagohvi, Meli, Jani, Susani, ale Maga\\
\index{What is your name?}
\index{ᎦᏙ ᏕᏣᏙᎠ?}
\chapter{What is your name? - ᎦᏙ ᏕᏣᏙᎠ?}
\index{What You Will Learn}\subsection{What You Will Learn}
In this unit you will learn:
\begin{itemize}
\item REMOVE
\end{itemize}\newpage

\subsection{Dialog - }
\begin{tabular}{p{2cm} p{11cm}}
not a valid letterᎡᎼᎥᎡ:\newline \textcolor{red}{REMOVE}: & not a valid letterᎡᎼᎥᎡ 
\newline\textcolor{red}{REMOVE}\\
not a valid letterᎡᎼᎥᎡ:\newline \textcolor{red}{REMOVE}: & not a valid letterᎡᎼᎥᎡ 
\newline\textcolor{red}{REMOVE}\\
not a valid letterᎡᎼᎥᎡ:\newline \textcolor{red}{REMOVE}: & not a valid letterᎡᎼᎥᎡ 
\newline\textcolor{red}{REMOVE}\\
not a valid letterᎡᎼᎥᎡ:\newline \textcolor{red}{REMOVE}: & not a valid letterᎡᎼᎥᎡ 
\newline\textcolor{red}{REMOVE}\\
not a valid letterᎡᎼᎥᎡ:\newline \textcolor{red}{REMOVE}: & not a valid letterᎡᎼᎥᎡ 
\newline\textcolor{red}{REMOVE}\\
not a valid letterᎡᎼᎥᎡ:\newline \textcolor{red}{REMOVE}: & not a valid letterᎡᎼᎥᎡ 
\newline\textcolor{red}{REMOVE}\\
not a valid letterᎡᎼᎥᎡ:\newline \textcolor{red}{REMOVE}: & not a valid letterᎡᎼᎥᎡ 
\newline\textcolor{red}{REMOVE}\\
not a valid letterᎡᎼᎥᎡ:\newline \textcolor{red}{REMOVE}: & not a valid letterᎡᎼᎥᎡ 
\newline\textcolor{red}{REMOVE}\\
not a valid letterᎡᎼᎥᎡ:\newline \textcolor{red}{REMOVE}: & not a valid letterᎡᎼᎥᎡ 
\newline\textcolor{red}{REMOVE}\\
not a valid letterᎡᎼᎥᎡ:\newline \textcolor{red}{REMOVE}: & not a valid letterᎡᎼᎥᎡ 
\newline\textcolor{red}{REMOVE}\\
\end{tabular}
\\
\\
\\
\noindent\begin{tabular}{p{2cm} p{11cm}}Mary: & Hello.  How are you?  My name is Mary.  What is your name?\\
Daniel: & I am fine.  My name is Daniel.  This is my friend.  His name is John.\\
Mary: & Hello.  My name is Mary.  What is your name?\\
Daniel: & Hi.  My name is daniel.  How are you?\\
Mary: & I am fine.  And you?\\
Daniel: & I am fine too.  Is everything ok with you? (Is everything fine?)\\
Mary: & Yes everything is fine.  I am happy that I have seen you.\\
Daniel: & I am happy to have seen you too.\\
Mary: & Who is that?\\
Daniel: & That is John.  His name is John.\\
\end{tabular}
\subsection{Vocabulary - ᏗᎧᏁᎢᏍᏗ 
}
\begin{tabular}{p{3cm} p{11cm}}
REMMOVE & not a valid letterᎡᎼᎥᎡ 
 \newline \textcolor{red}{REMOVE}\\
\end{tabular}


    Meeting people pp 2-3 (name, to want)

    Simple questions pp4 (tsu, sgo, sg, s) and pp 74 Smith

    turn these statements into questions
    turn these questions into statements
    ask for xyz
    do you want xyz
    tell your friend you would like an apple
    Identify other people you see that you are not currently talking to.

    More on this in Chapter 4

    Is your name bob?
    Is his name Barry?
    \index{Numbers}
\index{ᏗᏎᏍᏗ}
\chapter{Numbers - ᏗᏎᏍᏗ}
\index{What You Will Learn}\subsection{What You Will Learn}
In this unit you will learn:
\begin{itemize}
\item Tell time
\end{itemize}\newpage

\subsection{Dialog - }
\begin{tabular}{p{2cm} p{11cm}}
not a valid letterᎡᎼᎥᎡ:\newline \textcolor{red}{REMOVE}: & not a valid letterᎡᎼᎥᎡ 
\newline\textcolor{red}{REMOVE}\\
\end{tabular}
\\
\\
\\
\noindent\begin{tabular}{p{2cm} p{11cm}}REMOVE: & REMOVE\\
\end{tabular}
\subsection{Vocabulary - ᏗᎧᏁᎢᏍᏗ 
}
\begin{tabular}{p{3cm} p{11cm}}
one & ᏌᏊ 
 \newline \textcolor{red}{saquu}\\
two & ᏔᎵ 
 \newline \textcolor{red}{tali}\\
three & ᏦᎢ 
 \newline \textcolor{red}{tsoi}\\
four & ᏅᎯᎩ 
 \newline \textcolor{red}{nvhigi}\\
five & ᎯᏍᎩ 
 \newline \textcolor{red}{hisgi}\\
six & ᏑᏓᎵ 
 \newline \textcolor{red}{sudali}\\
seven & ᎦᎵᏉᎩ 
 \newline \textcolor{red}{galiquogi}\\
eight & ᏣᏁᎳ 
 \newline \textcolor{red}{tsanela}\\
nine & ᏐᏁᎳ 
 \newline \textcolor{red}{sonela}\\
ten & ᏍᎪᎯ 
 \newline \textcolor{red}{sgohi}\\
first & ᎢᎬᏱᎢ 
 \newline \textcolor{red}{igvyii}\\
second & ᏔᎵᏁᎢ 
 \newline \textcolor{red}{talinei}\\
third & ᏦᎢᏁᎢ 
 \newline \textcolor{red}{tsoinei}\\
fourth & ᏅᏍᎩᏁᎢ 
 \newline \textcolor{red}{nvsginei}\\
fifth & ᎯᏍᎩᏁᎢ 
 \newline \textcolor{red}{hisginei}\\
sixth & ᏑᏓᎵᏁᎢ 
 \newline \textcolor{red}{sudalinei}\\
seventh & ᎦᎵᏉᎩᏁᎢ 
 \newline \textcolor{red}{galiquoginei}\\
eighth & ᏧᏁᎵᏁᎢ 
 \newline \textcolor{red}{tsunelinei}\\
ninth & ᏐᏁᎵᏁᎢ 
 \newline \textcolor{red}{sonelinei}\\
tenth & ᏍᎪᎯᏁᎢ 
 \newline \textcolor{red}{sgohinei}\\
\end{tabular}

\index{Cardinal Numbers}\subsection{Cardinal Numbers}
Cardinal Numbers are any of the numbers that express amount, as one, two, three,  etc. (distinguished from ordinal number).\cite{cardinalNumbers}\\
\\
Cardinal numbers answer the question: How many are there? and tell the total.\index{Ordinal Numbers}\subsection{Ordinal Numbers}
Cardinal numbers are any of the numbers that express degree, quality, or position in a series, as first, second, and third  (distinguished from cardinal number ).\cite{ordinalNumbers}\\
\\
Ordinal numbers answer the question: Where does it fit in a numbered set? and tell the order.\\
\\
Ord(inal) - Ord(er)\index{Months}
\index{}
\chapter{Months - }
\index{What You Will Learn}\subsection{What You Will Learn}
In this unit you will learn:
\begin{itemize}
\item REMOVE
\end{itemize}\newpage

\subsection{Dialog - }
\begin{tabular}{p{2cm} p{11cm}}
not a valid letterᎡᎼᎥᎡ:\newline \textcolor{red}{REMOVE}: & not a valid letterᎡᎼᎥᎡ 
\newline\textcolor{red}{REMOVE}\\
\end{tabular}
\\
\\
\\
\noindent\begin{tabular}{p{2cm} p{11cm}}REMOVE: & REMOVE\\
\end{tabular}
\subsection{Vocabulary - ᏗᎧᏁᎢᏍᏗ 
}
\begin{tabular}{p{3cm} p{11cm}}
January & ᎤᏃᎸᏔᏂ 
 \newline \textcolor{red}{Unolvtani}\\
February & ᎧᎦᎵ 
 \newline \textcolor{red}{Kagali}\\
March & ᎠᏅᏱ 
 \newline \textcolor{red}{Anvyi}\\
April & ᎧᏬᏂ 
 \newline \textcolor{red}{Kawoni}\\
May & ᎠᎾᏍᎬᏘ 
 \newline \textcolor{red}{Anasgvti}\\
June & ᏕᎭᎷᏱ 
 \newline \textcolor{red}{Dehaluyi}\\
July & ᎫᏰᏉᏂ 
 \newline \textcolor{red}{Guyegwoni}\\
August & ᎦᎶᏂ 
 \newline \textcolor{red}{Galoni}\\
September & ᏚᎵᏍᏗ 
 \newline \textcolor{red}{Dulisdi}\\
October & ᏚᏂᏂᏗ 
 \newline \textcolor{red}{Duninidi}\\
November & ᏅᏓᏕᏆ 
 \newline \textcolor{red}{Nvdadegwa}\\
December & ᎥᏍᎩᏱ 
 \newline \textcolor{red}{Vskiyi}\\
\end{tabular}

\footnote{Discussed in the section \hyperref[sec:daysOfWeekMeaning]{Days Of Week Meanings}}\index{Dates}
\index{}
\chapter{Dates - }
\index{What You Will Learn}\subsection{What You Will Learn}
In this unit you will learn:
\begin{itemize}
\item REMOVE
\end{itemize}\newpage

\subsection{Dialog - }
\begin{tabular}{p{2cm} p{11cm}}
not a valid letterᎡᎼᎥᎡ:\newline \textcolor{red}{REMOVE}: & not a valid letterᎡᎼᎥᎡ 
\newline\textcolor{red}{REMOVE}\\
\end{tabular}
\\
\\
\\
\noindent\begin{tabular}{p{2cm} p{11cm}}REMOVE: & REMOVE\\
\end{tabular}
\subsection{Vocabulary - ᏗᎧᏁᎢᏍᏗ 
}
\begin{tabular}{p{3cm} p{11cm}}
Monday & ᎤᎾᏙᏓᏉᏅᎢ 
 \newline \textcolor{red}{Unadodagwonvi}\\
Tuesday & ᏔᎵᏁ ᎢᎦ 
 \newline \textcolor{red}{Taline iga}\\
Wednesday & ᏦᎢᏁ ᎢᎦ 
 \newline \textcolor{red}{Joine iga}\\
Thursday & ᏅᎩᏁ ᎢᎦ 
 \newline \textcolor{red}{Nvhgine iga}\\
Friday & ᏧᏅᎩᎶᏍᏗ 
 \newline \textcolor{red}{Junvgilosdi}\\
Saturday & ᎤᎾᏙᏓᏈᏕᎾ 
 \newline \textcolor{red}{Unadodagwidena}\\
Sunday & ᎤᎾᏙᏓᏆᏍᎬᎢ 
 \newline \textcolor{red}{Unadodagwasgvi}\\
\end{tabular}

\footnote{Discussed in the section \hyperref[sec:daysOfWeekMeaning]{Days Of Week Meanings}}\index{Time, Counting, Hours, Minutes, Seconds, Fractions}
\index{}
\chapter{Time, Counting, Hours, Minutes, Seconds, Fractions - }
\index{What You Will Learn}\subsection{What You Will Learn}
In this unit you will learn:
\begin{itemize}
\item REMOVE
\end{itemize}\newpage

\subsection{Dialog - }
\begin{tabular}{p{2cm} p{11cm}}
not a valid letterᎡᎼᎥᎡ:\newline \textcolor{red}{REMOVE}: & not a valid letterᎡᎼᎥᎡ 
\newline\textcolor{red}{REMOVE}\\
\end{tabular}
\\
\\
\\
\noindent\begin{tabular}{p{2cm} p{11cm}}REMOVE: & REMOVE\\
\end{tabular}
\subsection{Vocabulary - ᏗᎧᏁᎢᏍᏗ 
}
\begin{tabular}{p{3cm} p{11cm}}
At what time? & ᎯᎳ ᎠᏟᎢᎵᏒ? 
 \newline \textcolor{red}{Hila atliilisv?}\\
What time is it? (what hour is it) & ᎯᎳ ᎢᏳᏩᏂᎸ? 
 \newline \textcolor{red}{Hila iyuwanilv?}\\
What time is it? & ᎯᎳ ᎠᏟᎢᎵ? 
 \newline \textcolor{red}{Hila atliili?}\\
When are you going? & ᎯᎳᏴ ᏖᏏ? 
 \newline \textcolor{red}{Hilayv tesi?}\\
It is 8:00 & ᏣᏁᎳ ᎢᏳᏩᏂᎸ. 
 \newline \textcolor{red}{Chanela iyuwanilv.}\\
1 Hour & ᏑᏟᎶᏓ 
 \newline \textcolor{red}{Sutliloda}\\
Hours & ᎢᏧᏟᎶᏓ 
 \newline \textcolor{red}{Ijutliloda}\\
Minutes & ᎢᏯᏔᏬᏍᏔᏅ 
 \newline \textcolor{red}{Iyatawostanv}\\
Seconds & ᎢᏯᏎᏢ 
 \newline \textcolor{red}{Iyasetlv}\\
Before & ᎤᏓᎷᎳ 
 \newline \textcolor{red}{Udalula}\\
After & ᎤᎶᏒᏍᏗ 
 \newline \textcolor{red}{Ulosvsdi}\\
When will it start? & ᎯᎳᏴ ᏛᏓᎴᏅᎯ? 
 \newline \textcolor{red}{Hilayv dvdalenvhi?}\\
When will it end? & ᎯᎳᏴ ᏛᎵᏍᏆᏗ? 
 \newline \textcolor{red}{Hilayv dvlisgwadi?}\\
9:45 (fifteen minutes before ten) & ᏍᎩᎦᏚ ᎢᏯᏔᏬᏍᏔᏅ ᎤᏓᎷᎳ ᏍᎪᎯ 
 \newline \textcolor{red}{Sgigadu iyatawostanv udalula sgohi}\\
10:15 (fifteen minutes after ten) & ᏍᎩᎦᏚ ᎢᏯᏔᏬᏍᏔᏅ ᎤᎶᏒᏍᏗ ᏍᎪᎯ 
 \newline \textcolor{red}{Sgigadu iyatawostanv ulosvsdi sgohi}\\
1:30 (one and a half) & ᏌᏊ ᎠᏰᏟ 
 \newline \textcolor{red}{Sagwu ayetli}\\
Today & ᎪᎯ ᎢᎦ 
 \newline \textcolor{red}{Gohi iga}\\
Tomorrow & ᏌᎾᎴ ᎢᏴ 
 \newline \textcolor{red}{Sanale iyv}\\
Morning & ᏌᎾᎴ ᏗᏜ 
 \newline \textcolor{red}{Sanale didla}\\
Yesterday & ᏒᎯ 
 \newline \textcolor{red}{Svhi}\\
Dawn & ᎤᎩᏥᏕᏱ 
 \newline \textcolor{red}{Ugitsideyi}\\
Afternoon & ᏒᎯᏰᏱ ᏗᏜ - 1 
 \newline \textcolor{red}{Svhiyeyi didla - 1}\\
Evening & ᏒᎯᏰᏱ - 2 
 \newline \textcolor{red}{Svhiyeyi - 2}\\
Night & ᎤᏒ 
 \newline \textcolor{red}{Usv}\\
Midnight & ᏒᏃᏱ - 3 
 \newline \textcolor{red}{Svnoyi - 3}\\
Day/ Noon & ᎢᎦ 
 \newline \textcolor{red}{Iga}\\
\end{tabular}

REMOVE\footnote{Any time after 12:00 p.m. until the sun starts to set.}\footnote{The time of day when the sun is setting.}\footnote{The time somewhere in the late time of night like 12:00 a.m.}\cite{walcpp47}\index{Colors}
\index{}
\chapter{Colors - }
\index{What You Will Learn}\subsection{What You Will Learn}
In this unit you will learn:
\begin{itemize}
\item REMOVE
\end{itemize}\newpage

\subsection{Dialog - }
\begin{tabular}{p{2cm} p{11cm}}
not a valid letterᎡᎼᎥᎡ:\newline \textcolor{red}{REMOVE}: & not a valid letterᎡᎼᎥᎡ 
\newline\textcolor{red}{REMOVE}\\
\end{tabular}
\\
\\
\\
\noindent\begin{tabular}{p{2cm} p{11cm}}REMOVE: & REMOVE\\
\end{tabular}
\subsection{Vocabulary - ᏗᎧᏁᎢᏍᏗ 
}
\begin{tabular}{p{3cm} p{11cm}}
REMMOVE & not a valid letterᎡᎼᎥᎡ 
 \newline \textcolor{red}{REMOVE}\\
\end{tabular}

\index{Shapes}
\index{}
\chapter{Shapes - }
\index{What You Will Learn}\subsection{What You Will Learn}
In this unit you will learn:
\begin{itemize}
\item REMOVE
\end{itemize}\newpage

\subsection{Dialog - }
\begin{tabular}{p{2cm} p{11cm}}
not a valid letterᎡᎼᎥᎡ:\newline \textcolor{red}{REMOVE}: & not a valid letterᎡᎼᎥᎡ 
\newline\textcolor{red}{REMOVE}\\
\end{tabular}
\\
\\
\\
\noindent\begin{tabular}{p{2cm} p{11cm}}REMOVE: & REMOVE\\
\end{tabular}
\subsection{Vocabulary - ᏗᎧᏁᎢᏍᏗ 
}
\begin{tabular}{p{3cm} p{11cm}}
REMMOVE & not a valid letterᎡᎼᎥᎡ 
 \newline \textcolor{red}{REMOVE}\\
Circle & ᎦᏐᏆᎸ 
 \newline \textcolor{red}{Gasogwalv}\\
Oval (long circle) & ᎦᏐᏆᎸ ᎦᏅᎯᏓ 
 \newline \textcolor{red}{Gasogwalv ganvhida}\\
Triangle (three sides) & ᏦᎢ ᏧᏅᏏᏱ 
 \newline \textcolor{red}{Joi junvsiyi}\\
Square & ᏅᎩ ᏧᏅᏏᏱ 
 \newline \textcolor{red}{Nvhgi junvsiyi}\\
Rectangle (long square) & ᎦᏅᎯᏓ ᏅᎩ ᏧᏅᏏᏱ 
 \newline \textcolor{red}{Ganvhida nvhgi junvsiyi}\\
Diamond & ᎪᏍᏓᏱ 
 \newline \textcolor{red}{Gosdayi}\\
Pentagon & ᎯᏍᎩ ᏧᏅᏏᏱ 
 \newline \textcolor{red}{Hisgi junvsiyi}\\
Star & ᏃᏈᏏ 
 \newline \textcolor{red}{Nokwisi}\\
Hexagon & ᏑᏓᎵ ᏧᏅᏏᏱ 
 \newline \textcolor{red}{Sudali junvsiyi}\\
Heptagon & ᎦᎵᏉᎩ ᏧᏅᏏᏱ 
 \newline \textcolor{red}{Galigwogi junvsiyi}\\
Octagon & ᏣᏁᎳ ᏧᏅᏏᏱ 
 \newline \textcolor{red}{Chanela junvsiyi}\\
Nonagon & ᏐᏁᎳ ᏧᏅᏏᏱ 
 \newline \textcolor{red}{Sonela junvsiyi}\\
Decagon & ᏍᎪᎯ ᏧᏅᏏᏱ 
 \newline \textcolor{red}{Sgohi junvsiyi}\\
\end{tabular}

\index{NOTES:}
\index{ᏓᏓᏚᎬ ᎪᏪᎵ 
}
\chapter{NOTES: - ᏓᏓᏚᎬ ᎪᏪᎵ 
}
\index{Dialect Breakdown}\subsection{Dialect Breakdown - ᎣᏔᎵ  ᎩᏚᏩ}
The Giduwah, or Eastern, dialect of Cherokee varies in some ways from the Otali, or Western, dialect dialect of Cherokee.  A simple example is ᎭᏩ (G) vs ᎰᏩ (O).  Different spellings, same word.  Both mean "ok, alright, sure".  The word "ᎰᏩ" is an affirmative response and can be understood to mean different things depending on how it is used. Two of the more common meanings are "Okay" and "You are welcome".\cite{joynerlesson4}

\label{sec:wordBreakdownTohiOsi}\section{Word Breakdown - ᏙᎯ and ᎣᏏ Tohi and Osi}Altman and Belt (pp91-92) have this to say about Tohi and Osi:Tohi is a Cherokee morpheme that indicates the state in which nature is flowing at its appropriate pace and everything is as it should be. This fundamental concept is used in greetings and responses (\textcolor{red}{Tohigwatsv?} and \textcolor{red}{Tohigwu.}), and in a variety of other instances and constructions that indicate an underlying concern with the notion that things be flowing well in the Cherokee world. Tohi can be glossed variously as "well," "peaceful," "unhurried," and "health." In the Cherokee speakers' view, if the state of tohi becomes disrupted there can be disastrous consequences, and communities that are disrupted in this way can be dangerous or unhealthy places to live.\\
\\
In addition to and as an adjunct to tohi, the concept of osi describes the proper state of the individual person. Visualized as upright, facing forward, and resting on a single point of balance, osi is also used in greetings and replies (\textcolor{red}{osigwatsv?} and \textcolor{red}{osigwu.}), and in other contexts that indicate that the notion of an individual’s state of being is crucial in ensuring that all is flowing well in the larger Cherokee world. Osi is properly understood as referring to the state of neutrality and balance, but it is most often glossed as "good." If individuals are out of balance, they can cause problems in the larger system.\cite{altmanBelt90-98}

\label{sec:daysOfWeekMeaning}\section{Word Breakdown - Notes on the meanings of the days of the week}Notes on the meanings of the days of the week:\cite{walc1pp46}\\
\textit{Unadodagwonvi} - When they have completed doing something all day\\
\textit{Ta’line iga} - The second day\\
\textit{Jo’ine iga} - The third day\\
\textit{Nvhgine iga} - The fourth day\\
\textit{Jun(v)gilosdi} - The day they wash their clothes\footnote{The first way to say Friday was actually "hisgine'iga" which means "the fifth day."}\\
\textit{Unadodagwidena} - The day before they do something all day (when you went to town)\\
\textit{Unadodagwasgv’i} - They day they do something all day.

%+Make Parts
%\part{Reader}
%-Make Parts
%\newpage
%\chapter{Reader -}
1. (Ꭳ)ᏏᏲ, ᎺᎵ. 2. (Ꭳ)ᏏᏲ, ᎹᎦ. 3. (Ꭳ)ᏏᏲ, ᏓᏂᎵ 4. (Ꭳ)ᏏᏲ, ᏑᏌᏂ 
\\
1. ᏙᏓᏓᎪᎲᎢ, ᎺᎵ ᎠᎴ ᏣᏂ. 2. ᏙᎾᏓᎪᎲᎢ, ᏓᏓᏏ 3. ᏙᎾᏓᎪᎲᎢ, ᏓᏂᎵ 4. ᏙᏓᏓᎪᎲᎢ, ᎺᎵ, ᏣᏂ, ᏑᏌᏂ, ᎠᎴ ᎹᎦ 
\\
\includepdf[pages={36-37,91-92,95, 108}]{C:/projects/GoogleDriveTimo/Cherokee Umbrella/cherokee/lessons/walc1.pdf}


%+Make Parts
%\part{Grammar}
%-Make Parts
%\includepdf[pages={66,68-79,82,103-105}]{C:/projects/GoogleDriveTimo/Cherokee Umbrella/cherokee/lessons/walc1.pdf}
\includepdf[pages={14-18, 20-27}]{C:/projects/GoogleDriveTimo/Cherokee Umbrella/cherokee/lessons/walc1.pdf}


%+Make Parts
%\part{Answers}
%-Make Parts
%\chapter{Answer Key -}


%+Make Parts
%\part{Appendices}
%-Make Parts

%\chapter{Appendix A - Charts}
%\includepdf[noautoscale,landscape, angle=-90, pages={1}]{/projects/GoogleDriveTimo/Cherokee Umbrella/charts/cnosite/3D Solar System.jpg}
%\includepdf[angle=90]{/projects/GoogleDriveTimo/Cherokee Umbrella/charts/cnosite/3DSolarSystem.jpg}

\includegraphics{/projects/GoogleDriveTimo/Cherokee Umbrella/charts/cnosite/3DSolarSystem.jpg}
%\includepdf{/projects/GoogleDriveTimo/Cherokee Umbrella/charts/cnosite/3DSolarSystem.jpg}
\includepdf{/projects/GoogleDriveTimo/Cherokee Umbrella/charts/cnosite/Africa.pdf}
\includepdf{/projects/GoogleDriveTimo/Cherokee Umbrella/charts/cnosite/Antarctica.pdf}
\includepdf{/projects/GoogleDriveTimo/Cherokee Umbrella/charts/cnosite/AsiaandPacific.pdf}
\includepdf{/projects/GoogleDriveTimo/Cherokee Umbrella/charts/cnosite/BodyParts.pdf}
\includepdf{/projects/GoogleDriveTimo/Cherokee Umbrella/charts/cnosite/Canada.pdf}
\includepdf{/projects/GoogleDriveTimo/Cherokee Umbrella/charts/cnosite/CentralandEasternEurope.pdf}
\includepdf{/projects/GoogleDriveTimo/Cherokee Umbrella/charts/cnosite/CommunitySpeakerEnglish.pdf}
\includepdf{/projects/GoogleDriveTimo/Cherokee Umbrella/charts/cnosite/CommunitySpeakerMap1.pdf}
\includepdf{/projects/GoogleDriveTimo/Cherokee Umbrella/charts/cnosite/Continents.pdf}
\includegraphics{/projects/GoogleDriveTimo/Cherokee Umbrella/charts/cnosite/DolphinDiagram.jpg}
\includepdf{/projects/GoogleDriveTimo/Cherokee Umbrella/charts/cnosite/Europe.pdf}
\includepdf{/projects/GoogleDriveTimo/Cherokee Umbrella/charts/cnosite/HumanBody.pdf}
\includepdf{/projects/GoogleDriveTimo/Cherokee Umbrella/charts/cnosite/HumanSkeleton.pdf}
\includegraphics{/projects/GoogleDriveTimo/Cherokee Umbrella/charts/cnosite/Individual50States.png}
\includepdf{/projects/GoogleDriveTimo/Cherokee Umbrella/charts/cnosite/MexicoandSouthAmerica.pdf}
\includepdf{/projects/GoogleDriveTimo/Cherokee Umbrella/charts/cnosite/MidEast.pdf}
\includepdf{/projects/GoogleDriveTimo/Cherokee Umbrella/charts/cnosite/OfficeOnline.pdf}
\includepdf{/projects/GoogleDriveTimo/Cherokee Umbrella/charts/cnosite/OklahomaCountiesMap.pdf}
\includepdf{/projects/GoogleDriveTimo/Cherokee Umbrella/charts/cnosite/Semedite.pdf}
\includegraphics{/projects/GoogleDriveTimo/Cherokee Umbrella/charts/cnosite/SolarSystem.jpg}
\includepdf{/projects/GoogleDriveTimo/Cherokee Umbrella/charts/cnosite/UnitedStates.pdf}
\includegraphics{/projects/GoogleDriveTimo/Cherokee Umbrella/charts/cnosite/WashingHands.jpg}
\includepdf{/projects/GoogleDriveTimo/Cherokee Umbrella/charts/cnosite/World16x29.pdf}

\chapter{Document Changelist}
\begin{tabular}{p{3cm} p{11cm}}
18Jun21 & Added weather, finished organizing chapters; added specificity for footnotes with external urls\\
15Jun21 & Adjusted more sections; added some support for asciidoc\\
13Jun21 & Added more pieces and organized more sections \\
10Jun21 & Added date section and times, organized table of content\\
8Jun21 & Added reader section, answer key, updated exercises to have English, phonetic, and syllabary displays \\
7Jun21 & Made adjustments for page rendering in HTML and XELATEX
\end{tabular}



%+Bibliography
\begin{thebibliography}{99}
\bibitem{cardinalNumbers} http://dictionary.reference.com/browse/cardinal+numbers?s=t
\bibitem{ordinalNumbers} http://dictionary.reference.com/browse/ordinal+numbers?s=t
\bibitem{walcpp47} walc pp47
\bibitem{joynerlesson4} Cherokee Lessons Michael Joyner
\bibitem{altmanBelt90-98} Altman, H.M., \& Belt, T.N. (2008). Reading History: Cherokee History through a Cherokee Lens. Native South 1, 90-98. http://doi.org/10.1353/nso.0.0003
\bibitem{walc1pp46} We Are Learning Cherokee pp46
\end{thebibliography}
%-Bibliography


%
%\includepdf[pages=43-45, trim=55 100 45 250, clip=true]{w:/cherokeeumbrella/cherokee/lessons/BeginningCherokeeSearchable04.pdf}
%\includepdf[pages=43-45]{w:/cherokeeumbrella/cherokee/lessons/BeginningCherokeeSearchable04.pdf}
%\chapter{CHAPTER2}
\chapter{CHAPTER 3}
\includepdf[pages={43-45, 84-88}]{w:/cherokeeumbrella/cherokee/lessons/BeginningCherokeeSearchable04.pdf}
\\cite{holmessmith3234}

%\chapter{CHAPTER 4}

\chapter{CHAPTER 5}
\includepdf[pages={102-105, 109-113, 132-136, 92-95, 118-125, 79,80, 151-157}]{w:/cherokeeumbrella/cherokee/lessons/BeginningCherokeeSearchable04.pdf}

\chapter{CHAPTER 6}
\includepdf[pages={16-18, 59-63, 39-43}]{w:/cherokeeumbrella/cherokee/lessons/BeginningCherokeeSearchable04.pdf}

\chapter{CHAPTER 7}
\includepdf[pages={47-50, 85}]{w:/cherokeeumbrella/cherokee/lessons/BeginningCherokeeSearchable04.pdf}

\chapter{CHAPTER 8}
\includepdf[pages={173,174, 129-137, 149-156, 159-168, 175-182}]{w:/cherokeeumbrella/cherokee/lessons/BeginningCherokeeSearchable04.pdf}

\chapter{CHAPTER 9}
\includepdf[pages={245-252}]{w:/cherokeeumbrella/cherokee/lessons/BeginningCherokeeSearchable04.pdf}

\chapter{CHAPTER 10}
\includepdf[pages={205-210, 213-223}]{w:/cherokeeumbrella/cherokee/lessons/BeginningCherokeeSearchable04.pdf}

\chapter{CHAPTER 11}
\includepdf[pages={226-233, 236-242}]{w:/cherokeeumbrella/cherokee/lessons/BeginningCherokeeSearchable04.pdf}

\chapter{CHAPTER 12}
\includepdf[pages={186-198, 325-330, 22-26}]{w:/cherokeeumbrella/cherokee/lessons/BeginningCherokeeSearchable04.pdf}

\cite{cherokeeNationDownloads}


%+MakeIndex
\printindex
%-MakeIndex

\end{document}