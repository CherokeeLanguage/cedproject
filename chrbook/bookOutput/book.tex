\index{Pronunciation and Syllabary}
\index{}
\chapter{Pronunciation and Syllabary - }
\includepdf[pages={9-10, 109}]{W:/GOOGLEDRIVE/Cherokee Umbrella/sort/timo/walc1.pdf}
\index{Greet Others}
\index{ᎤᎾᎵᎮᎵᏤᏘ}
\chapter{Greet Others - ᎤᎾᎵᎮᎵᏤᏘ}
\index{What You Will Learn}\subsection{What You Will Learn}
In this unit you will learn:
\begin{itemize}
\item How to greet people
\item How to make short descriptive phrases
\item About definites and indefinites
\item Say "hello" and "goodbye"
\item Ask how someone is doing
\end{itemize}\newpage

\newpage\subsection{Dialog - ᎠᎾᎵᏃᎮᏍᎬ}
\begin{tabular}{p{2cm} p{11cm}}
ᏓᏂᎵ:\newline \textcolor{red}{Danili}: & ᎣᏏᏲ.  ᏓᏂᎵ ᏓᏩᏙᎠ.  ᎦᏙ ᏕᏣᏙᎠ? 
\newline\textcolor{red}{Osiyo.  Danili dawadoa.  Gado dejadoa?}\\
ᏑᏌᏂ:\newline \textcolor{red}{Susani}: & ᎣᏏᏲ.  ᏑᏌᏂ ᏓᏩᏙᎠ.  ᏙᎯᏧ? 
\newline\textcolor{red}{Osiyo.  Susani dawadoa.  Dohiju?}\\
ᏓᏂᎵ:\newline \textcolor{red}{Danili}: & ᎣᏏᏊ.  ᏂᎯᎾᎲ? 
\newline\textcolor{red}{Osigwu.  Nihinahv?}\\
ᏑᏌᏂ:\newline \textcolor{red}{Susani}: & ᎣᏍᏓ.  ᏙᎾᏓᎪᎲᎢ. 
\newline\textcolor{red}{Osda.  Donadagohvi.}\\
ᏓᏂᎵ:\newline \textcolor{red}{Danili}: & ᏙᎾᏓᎪᎲᎢ. 
\newline\textcolor{red}{Donadagohvi.}\\
\end{tabular}
\\
\\
\\
\noindent\begin{tabular}{p{2cm} p{11cm}}Daniel: & Hello.  My name is Daniel.  What is your name?\\
Susan: & Hello.  My name is Susan.  How are you?\\
Daniel: & I'm fine.  And you?\\
Susan: & Good.  Goodbye.\\
Daniel: & Goodbye.\\
\end{tabular}
\vfill\newpage\subsection{Vocabulary - ᏗᎧᏁᎢᏍᏗ 
}
\begin{minipage}{\linewidth}
\begin{tabular}{p{5cm} p{9cm}}
Titus & ᏓᏓᏏ 
 \newline \textcolor{red}{Dadasi}\\
Timothy & ᏗᎹᏗ 
 \newline \textcolor{red}{Dimadi}\\
Mary & ᎺᎵ 
 \newline \textcolor{red}{Meli}\\
Mark & ᎹᎦ 
 \newline \textcolor{red}{Maga}\\
Daniel & ᏓᏂᎵ 
 \newline \textcolor{red}{Danili}\\
Susan & ᏑᏌᏂ 
 \newline \textcolor{red}{Susani}\\
John & ᏣᏂ 
 \newline \textcolor{red}{Jani}\\
Hello & (Ꭳ)ᏏᏲ 
 \newline \textcolor{red}{(o)siyo}\\
And & ᎠᎴ 
  or ᏃᎴ 
 \newline \textcolor{red}{ale <e>or nole}\\
Good & ᎣᏍᏓ 
 \newline \textcolor{red}{osda}\\
Well/Fine & ᎣᏏᏊ 
 \newline \textcolor{red}{osigwu}\\
\end{tabular}
\end{minipage}

\
\index{Hello}\subsection{Hello - ᎣᏏᏲ}
In Cherokee there is only one way to say 'Good Morning,' 'Good Afternoon,' 'Good Evening,' and Hello; that is by saying ᎣᏏᏲ \textcolor{red}{(o)siyo} or the shortened version ᏏᏲ \textcolor{red}{siyo}.\footnote{We will follow the convention of placing optional syllables in parenthesis.  You will see this written as (O)siyo.  The parentheses around the 'O' mean that the voicing of the 'O' is optional.}\footnote{Osi will be discussed more in the section \hyperref[sec:wordBreakdownTohiOsi]{Word Breakdown - Tohi and Osi}}
\section{Exercise - ᎠᎵᏏᎾᎯᏍᏗᏍᎬ ᏗᎬᏙᏗ}
Translate to Cherokee syllabary and the phonetic equivalent\\
1. Hello, Mary 2. Hello, Mark 3. Hello, Daniel 4. Hello, Susan\\
1. (O)siyo, Meli. 2. (O)siyo, Maga. 3. (O)siyo, Danili 4. (O)siyo, Susani\\
\
\index{Goodbye}\subsection{Goodbye - ᏙᎾᏓᎪᎲᎢ}
There is no word for 'goodbye' only 'to meet again'. The way to say 'goodbye' to one person is ᏙᎾᏓᎪᎲᎢ \textcolor{red}{donadagohvi}. If you would like to say 'goodbye' to more than one person you would say ᏙᏓᏓᎪᎲᎢ \textcolor{red}{dodadagohvi}.  Lit: Let's meet again.\footnote{We will discuss the plurality prefixes (d-) in the section \hyperref[sec:wordBreakdownPluralityPrefixes]{Word Breakdown - Plurality Prefixes}}
\section{Exercise - ᎠᎵᏏᎾᎯᏍᏗᏍᎬ ᏗᎬᏙᏗ}
Translate to Cherokee syllabary and the phonetic equivalent\\
1. Goodbye, Mary and John. 2. Goodbye, Titus. 3. Goodbye, Daniel. 4. Goodbye, Mary, John, Susan, and Mark.\\
1. Dodadagohvi, Meli ale Jani. 2. Donadagohvi, Dadasi 3. Donadagohvi, Danili 4. Dodadagohvi, Meli, Jani, Susani, ale Maga\\
\includepdf[pages={27}, trim=15 100 5 305,  clip=true]{W:/GOOGLEDRIVE/Cherokee Umbrella/OtherLanguageReference/132903393-Teach-Yourself-Arabic.pdf}
