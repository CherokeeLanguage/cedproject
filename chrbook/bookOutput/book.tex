\index{Greet Others}
\index{ᎤᎾᎵᎮᎵᏤᏘ}
\chapter{Greet Others - ᎤᎾᎵᎮᎵᏤᏘ}
\index{What You Will Learn}\subsection{What You Will Learn}
In this unit you will learn:
\begin{itemize}
\item How to greet people
\item How to make short descriptive phrases
\item About definites and indefinites
\item Say "hello" and "goodbye"
\item Ask how someone is doing
\end{itemize}\newpage

\subsection{Dialog - }
\begin{tabular}{p{2cm} p{11cm}}
ᏓᏂᎵ:\newline \textcolor{red}{Danili}: & ᎣᏏᏲ.  ᏓᏂᎵ ᏓᏩᏙᎠ.  ᎦᏙ ᏕᏣᏙᎠ? 
\newline\textcolor{red}{Osiyo.  Danili dawadoa.  Gado dejadoa?}\\
ᏑᏌᏂ:\newline \textcolor{red}{Susani}: & ᎣᏏᏲ.  ᏑᏌᏂ ᏓᏩᏙᎠ.  ᏙᎯᏧ? 
\newline\textcolor{red}{Osiyo.  Susani dawadoa.  Dohiju?}\\
ᏓᏂᎵ:\newline \textcolor{red}{Danili}: & ᎣᏏᏊ.  ᏂᎯᎾᎲ? 
\newline\textcolor{red}{Osigwu.  Nihinahv?}\\
ᏑᏌᏂ:\newline \textcolor{red}{Susani}: & ᎣᏍᏓ.  ᏙᎾᏓᎪᎲᎢ. 
\newline\textcolor{red}{Osda.  Donadagohvi.}\\
ᏓᏂᎵ:\newline \textcolor{red}{Danili}: & ᏙᎾᏓᎪᎲᎢ. 
\newline\textcolor{red}{Donadagohvi.}\\
\end{tabular}
\\
\\
\\
\noindent\begin{tabular}{p{2cm} p{11cm}}Daniel: & Hello.  My name is Daniel.  What is your name?\\
Susan: & Hello.  My name is Susan.  How are you?\\
Daniel: & I'm fine.  And you?\\
Susan: & Good.  Goodbye.\\
Daniel: & Goodbye.\\
\end{tabular}
\subsection{Vocabulary - ᏗᎧᏁᎢᏍᏗ 
}
\begin{tabular}{p{3cm} p{11cm}}
Titus & ᏓᏓᏏ 
 \newline \textcolor{red}{Dadasi}\\
Timothy & ᏗᎹᏗ 
 \newline \textcolor{red}{Dimadi}\\
Mary & ᎺᎵ 
 \newline \textcolor{red}{Meli}\\
Mark & ᎹᎦ 
 \newline \textcolor{red}{Maga}\\
Daniel & ᏓᏂᎵ 
 \newline \textcolor{red}{Danili}\\
Susan & ᏑᏌᏂ 
 \newline \textcolor{red}{Susani}\\
John & ᏣᏂ 
 \newline \textcolor{red}{Jani}\\
Hello & (Ꭳ)ᏏᏲ 
 \newline \textcolor{red}{(o)siyo}\\
And & ᎠᎴ 
  or ᏃᎴ 
 \newline \textcolor{red}{ale <e>or nole}\\
Good & ᎣᏍᏓ 
 \newline \textcolor{red}{osda}\\
Well/Fine & ᎣᏏᏊ 
 \newline \textcolor{red}{osigwu}\\
\end{tabular}

\index{What is your name?}
\index{ᎦᏙ ᏕᏣᏙᎠ?}
\chapter{What is your name? - ᎦᏙ ᏕᏣᏙᎠ?}
\index{What You Will Learn}\subsection{What You Will Learn}
In this unit you will learn:

\subsection{Dialog - }
\noindent\begin{tabular}{p{2cm} p{11cm}}Mary: & Hello.  How are you?  My name is Mary.  What is your name?\\
Daniel: & I am fine.  My name is Daniel.  This is my friend.  His name is John.\\
Mary: & Hello.  My name is Mary.  What is your name?\\
Daniel: & Hi.  My name is daniel.  How are you?\\
Mary: & I am fine.  And you?\\
Daniel: & I am fine too.  Is everything ok with you? (Is everything fine?)\\
Mary: & Yes everything is fine.  I am happy that I have seen you.\\
Daniel: & I am happy to have seen you too.\\
Mary: & Who is that?\\
Daniel: & That is John.  His name is John.\\
\end{tabular}
\subsection{Vocabulary - ᏗᎧᏁᎢᏍᏗ 
}
\begin{tabular}{p{3cm} p{11cm}}
\end{tabular}

\index{Numbers}
\index{ᏗᏎᏍᏗ}
\chapter{Numbers - ᏗᏎᏍᏗ}
\index{What You Will Learn}\subsection{What You Will Learn}
In this unit you will learn:
\begin{itemize}
\item REMOVE
\end{itemize}\newpage

\subsection{Dialog - }

\noindent\begin{tabular}{p{2cm} p{11cm}}\end{tabular}
\subsection{Vocabulary - ᏗᎧᏁᎢᏍᏗ 
}
\begin{tabular}{p{3cm} p{11cm}}
one & ᏌᏊ 
 \newline \textcolor{red}{saquu}\\
two & ᏔᎵ 
 \newline \textcolor{red}{tali}\\
three & ᏦᎢ 
 \newline \textcolor{red}{tsoi}\\
four & ᏅᎯᎩ 
 \newline \textcolor{red}{nvhigi}\\
five & ᎯᏍᎩ 
 \newline \textcolor{red}{hisgi}\\
six & ᏑᏓᎵ 
 \newline \textcolor{red}{sudali}\\
seven & ᎦᎵᏉᎩ 
 \newline \textcolor{red}{galiquogi}\\
eight & ᏣᏁᎳ 
 \newline \textcolor{red}{tsanela}\\
nine & ᏐᏁᎳ 
 \newline \textcolor{red}{sonela}\\
ten & ᏍᎪᎯ 
 \newline \textcolor{red}{sgohi}\\
first & ᎢᎬᏱᎢ 
 \newline \textcolor{red}{igvyii}\\
second & ᏔᎵᏁᎢ 
 \newline \textcolor{red}{talinei}\\
third & ᏦᎢᏁᎢ 
 \newline \textcolor{red}{tsoinei}\\
fourth & ᏅᏍᎩᏁᎢ 
 \newline \textcolor{red}{nvsginei}\\
fifth & ᎯᏍᎩᏁᎢ 
 \newline \textcolor{red}{hisginei}\\
sixth & ᏑᏓᎵᏁᎢ 
 \newline \textcolor{red}{sudalinei}\\
seventh & ᎦᎵᏉᎩᏁᎢ 
 \newline \textcolor{red}{galiquoginei}\\
eighth & ᏧᏁᎵᏁᎢ 
 \newline \textcolor{red}{tsunelinei}\\
ninth & ᏐᏁᎵᏁᎢ 
 \newline \textcolor{red}{sonelinei}\\
tenth & ᏍᎪᎯᏁᎢ 
 \newline \textcolor{red}{sgohinei}\\
\end{tabular}

\index{Dialect Breakdown}\section{Dialect Breakdown - ᎣᏔᎵ - ᎩᏚᏩ}
The Giduwah, or Eastern, dialect of Cherokee varies in some ways from the Otali, or Western, dialect dialect of Cherokee.  A simple example is ᎭᏩ (G) vs ᎰᏩ (O).  Different spellings, same word.  Both mean "ok, alright, sure".  The word "ᎰᏩ" is an affirmative response and can be understood to mean different things depending on how it is used. Two of the more common meanings are "Okay" and "You are welcome".\cite{joynerlesson4}\label{sec:wordBreakdownTohiOsi}\section{Word Breakdown - ᏙᎯ and ᎣᏏ Tohi and Osi}Altman and Belt (pp91-92) have this to say about Tohi and Osi:Tohi is a Cherokee morpheme that indicates the state in which nature is flowing at its appropriate pace and everything is as it should be. This fundamental concept is used in greetings and responses (\textcolor{red}{Tohigwatsv?} and \textcolor{red}{Tohigwu.}), and in a variety of other instances and constructions that indicate an underlying concern with the notion that things be flowing well in the Cherokee world. Tohi can be glossed variously as "well," "peaceful," "unhurried," and "health." In the Cherokee speakers' view, if the state of tohi becomes disrupted there can be disastrous consequences, and communities that are disrupted in this way can be dangerous or unhealthy places to live.

In addition to and as an adjunct to tohi, the concept of osi describes the proper state of the individual person. Visualized as upright, facing forward, and resting on a single point of balance, osi is also used in greetings and replies (\textcolor{red}{osigwatsv?} and \textcolor{red}{osigwu.}, and in other contexts that indicate that the notion of an individual’s state of being is crucial in ensuring that all is flowing well in the larger Cherokee world. Osi is properly understood as referring to the state of neutrality and balance, but it is most often glossed as "good." If individuals are out of balance, they can cause problems in the larger system.\cite{altmanBelt90-98}