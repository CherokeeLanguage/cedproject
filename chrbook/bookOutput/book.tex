\begin{multicols}
\index{Pronunciation and Syllabary}
\index{}
\chapter{Pronunciation and Syllabary - }
\includepdf[pages={9-10, 109}]{W:/GOOGLEDRIVE/Cherokee Umbrella/sort/timo/walc1.pdf}
\index{Greet Others}
\index{ᎤᎾᎵᎮᎵᏤᏘ}
\chapter{Greet Others - ᎤᎾᎵᎮᎵᏤᏘ}
\index{What You Will Learn}\subsection{What You Will Learn}
In this unit you will learn:
\begin{itemize}
\item How to greet people
\item How to make short descriptive phrases
\item About definites and indefinites
\item Say "hello" and "goodbye"
\item Ask how someone is doing
\end{itemize}\newpage

\newpage\subsection{Dialog - ᎠᎾᎵᏃᎮᏍᎬ}
\begin{tabular}{p{2cm} p{11cm}}
ᏓᏂᎵ:\newline \textcolor{red}{Danili}: & ᎣᏏᏲ.  ᏓᏂᎵ ᏓᏩᏙᎠ.  ᎦᏙ ᏕᏣᏙᎠ? 
\newline\textcolor{red}{Osiyo.  Danili dawadoa.  Gado dejadoa?}\\
ᏑᏌᏂ:\newline \textcolor{red}{Susani}: & ᎣᏏᏲ.  ᏑᏌᏂ ᏓᏩᏙᎠ.  ᏙᎯᏧ? 
\newline\textcolor{red}{Osiyo.  Susani dawadoa.  Dohiju?}\\
ᏓᏂᎵ:\newline \textcolor{red}{Danili}: & ᎣᏏᏊ.  ᏂᎯᎾᎲ? 
\newline\textcolor{red}{Osigwu.  Nihinahv?}\\
ᏑᏌᏂ:\newline \textcolor{red}{Susani}: & ᎣᏍᏓ.  ᏙᎾᏓᎪᎲᎢ. 
\newline\textcolor{red}{Osda.  Donadagohvi.}\\
ᏓᏂᎵ:\newline \textcolor{red}{Danili}: & ᏙᎾᏓᎪᎲᎢ. 
\newline\textcolor{red}{Donadagohvi.}\\
\end{tabular}
\\
\\
\\
\noindent\begin{tabular}{p{2cm} p{11cm}}Daniel: & Hello.  My name is Daniel.  What is your name?\\
Susan: & Hello.  My name is Susan.  How are you?\\
Daniel: & I'm fine.  And you?\\
Susan: & Good.  Goodbye.\\
Daniel: & Goodbye.\\
\end{tabular}
\vfill\newpage\subsection{Vocabulary - ᏗᎧᏁᎢᏍᏗ 
}
\begin{minipage}{\linewidth}
\begin{tabular}{p{5cm} p{9cm}}
Titus & ᏓᏓᏏ 
 \newline \textcolor{red}{Dadasi}\\
Timothy & ᏗᎹᏗ 
 \newline \textcolor{red}{Dimadi}\\
Mary & ᎺᎵ 
 \newline \textcolor{red}{Meli}\\
Mark & ᎹᎦ 
 \newline \textcolor{red}{Maga}\\
Daniel & ᏓᏂᎵ 
 \newline \textcolor{red}{Danili}\\
Susan & ᏑᏌᏂ 
 \newline \textcolor{red}{Susani}\\
John & ᏣᏂ 
 \newline \textcolor{red}{Jani}\\
Hello & (Ꭳ)ᏏᏲ 
 \newline \textcolor{red}{(o)siyo}\\
And & ᎠᎴ 
  or ᏃᎴ 
 \newline \textcolor{red}{ale <e>or nole}\\
Good & ᎣᏍᏓ 
 \newline \textcolor{red}{osda}\\
Well/Fine & ᎣᏏᏊ 
 \newline \textcolor{red}{osigwu}\\
\end{tabular}
\end{minipage}

\
\index{Hello}\subsection{Hello - ᎣᏏᏲ}
In Cherokee there is only one way to say 'Good Morning,' 'Good Afternoon,' 'Good Evening,' and Hello; that is by saying ᎣᏏᏲ \textcolor{red}{(o)siyo} or the shortened version ᏏᏲ \textcolor{red}{siyo}.\footnote{We will follow the convention of placing optional syllables in parenthesis.  You will see this written as (O)siyo.  The parentheses around the 'O' mean that the voicing of the 'O' is optional.}\footnote{Osi will be discussed more in the section \hyperref[sec:wordBreakdownTohiOsi]{Word Breakdown - Tohi and Osi}}
\section{Exercise - ᎠᎵᏏᎾᎯᏍᏗᏍᎬ ᏗᎬᏙᏗ}
Translate to Cherokee syllabary and the phonetic equivalent\\
1. Hello, Mary 2. Hello, Mark 3. Hello, Daniel 4. Hello, Susan\\
1. (O)siyo, Meli. 2. (O)siyo, Maga. 3. (O)siyo, Danili 4. (O)siyo, Susani\\
\
\index{Goodbye}\subsection{Goodbye - ᏙᎾᏓᎪᎲᎢ}
There is no word for 'goodbye' only 'to meet again'. The way to say 'goodbye' to one person is ᏙᎾᏓᎪᎲᎢ \textcolor{red}{donadagohvi}. If you would like to say 'goodbye' to more than one person you would say ᏙᏓᏓᎪᎲᎢ \textcolor{red}{dodadagohvi}.  Lit: Let's meet again.\footnote{We will discuss the plurality prefixes (d-) in the section \hyperref[sec:wordBreakdownPluralityPrefixes]{Word Breakdown - Plurality Prefixes}}
\section{Exercise - ᎠᎵᏏᎾᎯᏍᏗᏍᎬ ᏗᎬᏙᏗ}
Translate to Cherokee syllabary and the phonetic equivalent\\
1. Goodbye, Mary and John. 2. Goodbye, Titus. 3. Goodbye, Daniel. 4. Goodbye, Mary, John, Susan, and Mark.\\
1. Dodadagohvi, Meli ale Jani. 2. Donadagohvi, Dadasi 3. Donadagohvi, Danili 4. Dodadagohvi, Meli, Jani, Susani, ale Maga\\
\includepdf[pages={27}, trim=15 100 5 305,  clip=true]{W:/GOOGLEDRIVE/Cherokee Umbrella/OtherLanguageReference/132903393-Teach-Yourself-Arabic.pdf}
\index{What is your name?}
\index{ᎦᏙ ᏕᏣᏙᎠ?}
\chapter{What is your name? - ᎦᏙ ᏕᏣᏙᎠ?}
\index{What You Will Learn}\subsection{What You Will Learn}
In this unit you will learn:
\begin{itemize}
\item Meeting people pp 2-3 (name, to want)
\item Simple questions pp4 (tsu, sgo, sg, s) and pp 74 Smith
\item turn these statements into questions
\item turn these questions into statements
\item ask for xyz
\item do you want xyz
\item tell your friend you would like an apple
\item Identify other people you see that you are not currently talking to.
\item More on this in Chapter 4
\item Is your name bob?
\item Is his name Barry?
\end{itemize}\newpage

\newpage\subsection{Dialog - ᎠᎾᎵᏃᎮᏍᎬ}
\begin{tabular}{p{2cm} p{11cm}}
\end{tabular}
\\
\\
\\
\noindent\begin{tabular}{p{2cm} p{11cm}}Mary: & Hello.  How are you?  My name is Mary.  What is your name?\\
Daniel: & I am fine.  My name is Daniel.  This is my friend.  His name is John.\\
Mary: & Hello.  My name is Mary.  What is your name?\\
Daniel: & Hi.  My name is daniel.  How are you?\\
Mary: & I am fine.  And you?\\
Daniel: & I am fine too.  Is everything ok with you? (Is everything fine?)\\
Mary: & Yes everything is fine.  I am happy that I have seen you.\\
Daniel: & I am happy to have seen you too.\\
Mary: & Who is that?\\
Daniel: & That is John.  His name is John.\\
\end{tabular}
\vfill\newpage\index{I'd like you to meet...}
\index{I'd like you to meet...}
\chapter{I'd like you to meet...}
[could be in What is your name?]\\
JAC I'd like you to meet\\
JAC What's new\\
JAC have you two met?\\
JAC glad to have met you\\
\index{I want, I see}
\index{I want, I see}
\chapter{I want, I see}
Deal with plurals and I we/want from Ch4\\
I see 2 apples\\
I want 4 bananas\\
I want some grapes\\
there are some verbs included in the book specifically for this purpose\\
\index{Where are you from?}
\index{}
\chapter{Where are you from? - }
\index{What You Will Learn}\subsection{What You Will Learn}
In this unit you will learn:
\begin{itemize}
\item ARC where are you from?
\item tell about oklahoma and nc
\item I am from xyz
\item pics of nationalities
\item reference Appendicies
\item ex10 where are these people from?
\item ex 12 list xyz of countries (area codes of states?)
\item where are you?  where are you from?
\item languages - german
\item countries - Germany
\item people - German
\item english - yonega (according-to-white)
\end{itemize}\newpage

\newpage\subsection{Vocabulary - ᏗᎧᏁᎢᏍᏗ 
}
\begin{minipage}{\linewidth}
\begin{tabular}{p{5cm} p{9cm}}
Where do you live? (1 person) & ᎭᏢ ᎯᏁᎳ? 
 \newline \textcolor{red}{Hadlv hinela?}\\
What is your address? (Where written-material it-you-get? & ᎭᏢ ᎪᏪᎵᎠ ᏗᏣᏁᏍᏗ? 
 \newline \textcolor{red}{Hadlv gowelia dijanesdi?}\\
My address is \underline{      }. (written-material it-I-get-at.) & ᎪᏪᎵ ᏗᎩᏁᏍᏗ    . 
 \newline \textcolor{red}{Goweli diginesdi    .}\\
people, tribe & ᏴᏫ 
 \newline \textcolor{red}{yvwi}\\
I am an Indian. (I originate of the real people.) & ᏥᏴᏫᏯᏛ. 
 \newline \textcolor{red}{Jiyvwiyadv.}\\
You are an Indian. & ᎯᏴᏫᏯᏛ. 
 \newline \textcolor{red}{hiyvwiyadv.}\\
Are you an Indian? & ᎯᏴᏫᏯᏍᎪᎲ? 
 \newline \textcolor{red}{Hiyvwiyasgohv?}\\
He is an Indian. & ᎠᏴᏫᏯᏛ. 
 \newline \textcolor{red}{Ayvwiyadv.}\\
What tribe of Indian does he belong to? & ᎦᏙ ᎤᏍᏗ ᎠᏴᏫᏯ? 
 \newline \textcolor{red}{Gado usdi ayvwiya?}\\
Do you speak English? & ᏳᏁᎦnot a valid letterᎡ ᎯᏬᏂᎠ? 
 \newline \textcolor{red}{Yunegake hiwonia?}\\
I am of Osage origin . (I originate as an ...) & ᎠᏌᏏ ᏥᏴᏫ. 
 \newline \textcolor{red}{Asasi jiyvwi.}\\
... as a Delaware & ᎠᏆᏅᎩ ᏥᏴᏫ. 
 \newline \textcolor{red}{Agwanvgi jiyvwi.}\\
... as a Creek & ᎠᎫᏌ ᏥᏴᏫ. 
 \newline \textcolor{red}{Agusa jiyvwi.}\\
... as a White & ᏥᏳᏁᎦ ᏥᏴᏫ. 
 \newline \textcolor{red}{Jiyunega jiyvwi.}\\
American Indian (yv-wi means a people, -ya means basic, real, true, essential) & ᎠᏴᏫᏯ 
 \newline \textcolor{red}{ayvwiya}\\
I speak Cherokee a little. (Little I-speak Cherokee) & ᎦᎣᏲᏟ ᏥᏬᏂᎠ ᏣᎳᎩ. 
 \newline \textcolor{red}{Gaoyotli jiwonia Jalagi.}\\
Do you speak Cherokee a little? & ᎦᏲᏟnot a valid letterᎡ ᎯᏬᏂᎠ ᏣᎳᎩ? 
 \newline \textcolor{red}{Gayotlike hiwonia Jalagi?}\\
He speaks Cherokee well. (Well-indeed Cherokee he-speaks.) & ᎣᏍᏓᏛ ᏣᎳᎩ ᎦᏬᏂᎠ. 
 \newline \textcolor{red}{Osdadv Jalagi gawonia.}\\
\end{tabular}
\end{minipage}

\index{Do you speak Tsalagi?}
\index{Do you speak Tsalagi?}
\chapter{Do you speak Tsalagi?}
Do you speak xyz?\\
JAC do you speak Japanese?\\
JAC please speak a little slower\\
other languages\\
\index{On the Telephone}
\index{}
\chapter{On the Telephone - }
\index{What You Will Learn}\subsection{What You Will Learn}
In this unit you will learn:
\begin{itemize}
\item ARC the telephone number is
\item what is your phone number?
\item ex11 what are the phone numbers and who do they belong to?
\item contacts list on your phone
\item email list?
\end{itemize}\newpage

\newpageMy telephone number is...their telephone number is...try number...is there a phone here?do you have a phone?the line is busythere is no answermay i speak with >>>>?speakingthis is titus speaking.can you go text bob?text bob and tell him I need...what is <name><your><my> number?\index{Address \& Email}
\index{}
\chapter{Address \& Email - }
\index{What You Will Learn}\subsection{What You Will Learn}
In this unit you will learn:
\begin{itemize}
\item JAC my address is
\item JAC writing and mailing letters
\item email - NOT JAC - but faxing didn't seem like a current topic
\end{itemize}\newpage

\newpageI live at <my address is....>where does .... live?He lives in this townshe lives at ....our address is....we live at...my room number is ...do you have a pen(cil)do you have a stamp?do you have an envelopewhere is the post officehow long will it take to get there?\index{Numbers}
\index{ᏗᏎᏍᏗ}
\chapter{Numbers - ᏗᏎᏍᏗ}
\index{What You Will Learn}\subsection{What You Will Learn}
In this unit you will learn:
\begin{itemize}
\item Tell time
\end{itemize}\newpage

\newpage\subsection{Vocabulary - ᏗᎧᏁᎢᏍᏗ 
}
\begin{minipage}{\linewidth}
\begin{tabular}{p{5cm} p{9cm}}
one & ᏌᏊ 
 \newline \textcolor{red}{saquu}\\
two & ᏔᎵ 
 \newline \textcolor{red}{tali}\\
three & ᏦᎢ 
 \newline \textcolor{red}{tsoi}\\
four & ᏅᎯᎩ 
 \newline \textcolor{red}{nvhigi}\\
five & ᎯᏍᎩ 
 \newline \textcolor{red}{hisgi}\\
six & ᏑᏓᎵ 
 \newline \textcolor{red}{sudali}\\
seven & ᎦᎵᏉᎩ 
 \newline \textcolor{red}{galiquogi}\\
eight & ᏣᏁᎳ 
 \newline \textcolor{red}{tsanela}\\
nine & ᏐᏁᎳ 
 \newline \textcolor{red}{sonela}\\
ten & ᏍᎪᎯ 
 \newline \textcolor{red}{sgohi}\\
Eleven & ᏌᏚ 
 \newline \textcolor{red}{Sadu}\\
Twelve & ᏔᎵᏚ 
 \newline \textcolor{red}{Talidu}\\
Thirteen & ᏦᎦᏚ 
 \newline \textcolor{red}{Jogadu}\\
Fourteen & ᏂᎦᏚ 
 \newline \textcolor{red}{Nigadu}\\
Fifteen & ᏍᎩᎦᏚ 
 \newline \textcolor{red}{Sgigadu}\\
Twenty one & ᏔᎵᏍᎪᎯ ᏌᏊ 
 \newline \textcolor{red}{Talisgohi sagwu}\\
Twenty two & ᏔᎵᏍᎪᎯ ᏔᎵ 
 \newline \textcolor{red}{Talisgohi tali}\\
Twenty three & ᏔᎵᏍᎪᎯ ᏦᎢ 
 \newline \textcolor{red}{Talisgohi joi}\\
Twenty four & ᏔᎵᏍᎪᎯ ᏅᎩ 
 \newline \textcolor{red}{Talisgohi nvhgi}\\
Twenty five & ᏔᎵᏍᎪᎯ ᎯᏍᎩ 
 \newline \textcolor{red}{Talisgohi hisgi}\\
\end{tabular}
\end{minipage}

\vfill\newpage\begin{minipage}{\linewidth}\begin{tabular}{p{5cm} p{9cm}}
Forty & ᏅᎩᏍᎪᎯ 
 \newline \textcolor{red}{Nvgisgohi}\\
Fifty & Ꭿnot a valid letterᏍᎪᎯ 
 \newline \textcolor{red}{Hiksgohi}\\
Sixty & ᏑᏓᎵᏍᎪᎯ 
 \newline \textcolor{red}{Sudalisgohi}\\
Seventy & ᎦᎵᏆᏍᎪᎯ 
 \newline \textcolor{red}{Galigwasgohi}\\
Six & ᏑᏓᎵ 
 \newline \textcolor{red}{Sudali}\\
Seven & ᎦᎵᏉᎩ 
 \newline \textcolor{red}{Galigwogi}\\
Eight & ᏣᏁᎳ 
 \newline \textcolor{red}{Chanela}\\
Nine & Ꮠnot a valid letterᏁᎳ 
 \newline \textcolor{red}{Sohnela}\\
Ten & ᏍᎪᎯ 
 \newline \textcolor{red}{Sgohi}\\
Sixteen & ᏓᎳᏚ 
 \newline \textcolor{red}{Daladu}\\
Seventeen & ᎦᎵᏆᏚ 
 \newline \textcolor{red}{Galigwadu}\\
Eighteen & ᏁᎳᏚ 
 \newline \textcolor{red}{Neladu}\\
Nineteen & Ꮠnot a valid letterᏁᎳᏚ 
 \newline \textcolor{red}{Sohneladu}\\
Twenty & ᏔᎵᏍᎪᎯ 
 \newline \textcolor{red}{Talisgohi}\\
Twenty six & ᏔᎵᏍᎪᎯ ᏑᏓᎵ 
 \newline \textcolor{red}{Talisgohi sudali}\\
Twenty seven & ᏔᎵᏍᎪᎯ ᎦᎵᏉᎩ 
 \newline \textcolor{red}{Talisgohi galigwogi}\\
Twenty eight & ᏔᎵᏍᎪᎯ ᏣᏁᎳ 
 \newline \textcolor{red}{Talisgohi chanela}\\
Twenty nine & ᏔᎵᏍᎪᎯ Ꮠnot a valid letterᏁᎳ 
 \newline \textcolor{red}{Talisgohi sohnela}\\
Thirty & ᏦᎢᏍᎪᎯ 
 \newline \textcolor{red}{Joisgohi}\\
Eighty & ᏁᎵᏍᎪᎯ 
 \newline \textcolor{red}{Nelisgohi}\\
\end{tabular}
\end{minipage}

\vfill\newpage\begin{minipage}{\linewidth}\begin{tabular}{p{5cm} p{9cm}}
Ninety & Ꮠnot a valid letterᏁᎵᏍᎪᎯ 
 \newline \textcolor{red}{Sohnelisgohi}\\
One Hundred & ᏍᎪᎯᏥᏆ 
 \newline \textcolor{red}{Sgohijigwa}\\
first & ᏌᏊᎯᏁ / ᎢᎬᏱᎢ 
 \newline \textcolor{red}{Sagwuhine / igvyii}\footnote{Both sagwuhine and igvyi mean first, but sagwuhine is more commonly used when expressing 21st, 31st, etc. -- We Are Learning Cherokee pp 41}\\
second & ᏔᎵᏁᎢ 
 \newline \textcolor{red}{talinei}\\
third & ᏦᎢᏁᎢ 
 \newline \textcolor{red}{tsoinei}\\
fourth & ᏅᏍᎩᏁᎢ 
 \newline \textcolor{red}{nvsginei}\\
fifth & ᎯᏍᎩᏁᎢ 
 \newline \textcolor{red}{hisginei}\\
sixth & ᏑᏓᎵᏁᎢ 
 \newline \textcolor{red}{sudalinei}\\
seventh & ᎦᎵᏉᎩᏁᎢ 
 \newline \textcolor{red}{galiquoginei}\\
eighth & ᏧᏁᎵᏁᎢ 
 \newline \textcolor{red}{tsunelinei}\\
ninth & ᏐᏁᎵᏁᎢ 
 \newline \textcolor{red}{sonelinei}\\
tenth & ᏍᎪᎯᏁᎢ 
 \newline \textcolor{red}{sgohinei}\\
11th & ᏌᏚᏏᏁ 
 \newline \textcolor{red}{Sadusine}\\
12th & ᏔᎵᏚᏏᏁ 
 \newline \textcolor{red}{Talidusine}\\
13th & ᏦᎦᏚᏏᏁ 
 \newline \textcolor{red}{Jogadusine}\\
14th & ᏂᎦᏚᏏᏁ 
 \newline \textcolor{red}{Nigadusine}\\
15th & ᏍᎩᎦᏚᏏᏁ 
 \newline \textcolor{red}{Sgigadusine}\\
16th & ᏓᎳᏚᏏᏁ 
 \newline \textcolor{red}{Daladusine}\\
17th & ᎦᎵᏆᏚᏏᏁ 
 \newline \textcolor{red}{Galigwadusine}\\
18th & ᏁᎳᏚᏏᏁ 
 \newline \textcolor{red}{Neladusine}\\
\end{tabular}
\end{minipage}

\vfill\newpage\begin{minipage}{\linewidth}\begin{tabular}{p{5cm} p{9cm}}
19th & Ꮠnot a valid letterᏁᎳᏚᏏᏁ 
 \newline \textcolor{red}{Sohneladusine}\\
20th & ᏔᎵᏍᎪᎯᏁ 
 \newline \textcolor{red}{Talisgohine}\\
21st & ᏔᎵᏍᎪᎢ ᏌᏊᎯᏁ 
 \newline \textcolor{red}{Talisgoi sagwuhine}\\
30th & ᏦᎢᏍᎪᎯᏁ 
 \newline \textcolor{red}{Joisgohine}\\
\end{tabular}
\end{minipage}

\
\index{Cardinal Numbers}\subsection{Cardinal Numbers}
Cardinal Numbers are any of the numbers that express amount, as one, two, three,  etc. (distinguished from ordinal number).\cite{cardinalNumbers}\\\\
Cardinal numbers answer the question: How many are there? and tell the total.\
\index{Ordinal Numbers}\subsection{Ordinal Numbers}
Ordinal numbers are any of the numbers that express degree, quality, or position in a series, as first, second, and third  (distinguished from cardinal number ).\cite{ordinalNumbers}\\\\
\noindent Ordinal numbers answer the question: Where does it fit in a numbered set? and tell the order.\\\\
\noindent Ord(inal) - Ord(er)\\\\
Ordinal numbers are very similar to the regular numbers in Cherokee. To make a number into an ordinal number, most of the time you will only add the suffix \textcolor{red}{-ne} to the end of the word. For some the suffix \textcolor{red}{-hine} (seen on 1st) and the suffix \textcolor{red}{-sine} (seen on 11th - 19th) needs to be added to change the number into an ordinal.\cite{walc141}counters1 cat2 pencilsanimate counterinanimate counter\index{Dates}
\index{}
\chapter{Dates - }
\subsection{Vocabulary - ᏗᎧᏁᎢᏍᏗ 
}
\begin{minipage}{\linewidth}
\begin{tabular}{p{5cm} p{9cm}}
Monday & ᎤᎾᏙᏓᏉᏅᎢ 
 \newline \textcolor{red}{Unadodagwonvi}\\
Tuesday & ᏔᎵᏁ ᎢᎦ 
 \newline \textcolor{red}{Taline iga}\\
Wednesday & ᏦᎢᏁ ᎢᎦ 
 \newline \textcolor{red}{Joine iga}\\
Thursday & ᏅᎩᏁ ᎢᎦ 
 \newline \textcolor{red}{Nvhgine iga}\\
Friday & ᏧᏅᎩᎶᏍᏗ 
 \newline \textcolor{red}{Junvgilosdi}\\
Saturday & ᎤᎾᏙᏓᏈᏕᎾ 
 \newline \textcolor{red}{Unadodagwidena}\\
Sunday & ᎤᎾᏙᏓᏆᏍᎬᎢ 
 \newline \textcolor{red}{Unadodagwasgvi}\\
\end{tabular}
\end{minipage}

\footnote{Discussed in the section \hyperref[sec:daysOfWeekMeaning]{Days Of Week Meanings}}the civil war was in 1861ny worlds fair took place in 1939\index{Time, Counting, Hours, Minutes, Seconds, Fractions}
\index{}
\chapter{Time, Counting, Hours, Minutes, Seconds, Fractions - }
\subsection{Vocabulary - ᏗᎧᏁᎢᏍᏗ 
}
\begin{minipage}{\linewidth}
\begin{tabular}{p{5cm} p{9cm}}
Hours & ᎢᏧᏟᎶᏓ 
 \newline \textcolor{red}{Ijutliloda}\\
Minutes & ᎢᏯᏔᏬᏍᏔᏅ 
 \newline \textcolor{red}{Iyatawostanv}\\
Seconds & ᎢᏯᏎᏢ 
 \newline \textcolor{red}{Iyasetlv}\\
Before & ᎤᏓᎷᎳ 
 \newline \textcolor{red}{Udalula}\\
After & ᎤᎶᏒᏍᏗ 
 \newline \textcolor{red}{Ulosvsdi}\\
Today & ᎪᎯ ᎢᎦ 
 \newline \textcolor{red}{Gohi iga}\\
Tomorrow & ᏌᎾᎴ ᎢᏴ 
 \newline \textcolor{red}{Sanale iyv}\\
Morning & ᏌᎾᎴ ᏗᏜ 
 \newline \textcolor{red}{Sanale didla}\\
Yesterday & ᏒᎯ 
 \newline \textcolor{red}{Svhi}\\
Dawn & ᎤᎩᏥᏕᏱ 
 \newline \textcolor{red}{Ugitsideyi}\\
Afternoon & ᏒᎯᏰᏱ ᏗᏜ 
 \newline \textcolor{red}{Svhiyeyi didla}\footnote{Any time after 12:00 p.m. until the sun starts to set.}\\
Evening & ᏒᎯᏰᏱ 
 \newline \textcolor{red}{Svhiyeyi}\footnote{The time of day when the sun is setting.}\\
Night & ᎤᏒ 
 \newline \textcolor{red}{Usv}\\
Midnight & ᏒᏃᏱ 
 \newline \textcolor{red}{Svnoyi}\footnote{The time somewhere in the late time of night like 12:00 a.m.}\\
Day/ Noon & ᎢᎦ 
 \newline \textcolor{red}{Iga}\\
At what time? & ᎯᎳ ᎠᏟᎢᎵᏒ? 
 \newline \textcolor{red}{Hila atliilisv?}\\
What time is it?\\ (what hour is it) & ᎯᎳ ᎢᏳᏩᏂᎸ? 
 \newline \textcolor{red}{Hila iyuwanilv?}\\
What time is it? & ᎯᎳ ᎠᏟᎢᎵ? 
 \newline \textcolor{red}{Hila atliili?}\\
When are you going? & ᎯᎳᏴ ᏖᏏ? 
 \newline \textcolor{red}{Hilayv tesi?}\\
It is 8:00 & ᏣᏁᎳ ᎢᏳᏩᏂᎸ. 
 \newline \textcolor{red}{Chanela iyuwanilv.}\\
\end{tabular}
\end{minipage}

\vfill\newpage\begin{minipage}{\linewidth}\begin{tabular}{p{5cm} p{9cm}}
1 Hour & ᏑᏟᎶᏓ 
 \newline \textcolor{red}{Sutliloda}\\
When will it start? & ᎯᎳᏴ ᏛᏓᎴᏅᎯ? 
 \newline \textcolor{red}{Hilayv dvdalenvhi?}\\
When will it end? & ᎯᎳᏴ ᏛᎵᏍᏆᏗ? 
 \newline \textcolor{red}{Hilayv dvlisgwadi?}\\
9:45 (fifteen minutes before ten) & ᏍᎩᎦᏚ ᎢᏯᏔᏬᏍᏔᏅ ᎤᏓᎷᎳ ᏍᎪᎯ 
 \newline \textcolor{red}{Sgigadu iyatawostanv udalula sgohi}\\
10:15 (fifteen minutes after ten) & ᏍᎩᎦᏚ ᎢᏯᏔᏬᏍᏔᏅ ᎤᎶᏒᏍᏗ ᏍᎪᎯ 
 \newline \textcolor{red}{Sgigadu iyatawostanv ulosvsdi sgohi}\\
1:30 (one and a half) & ᏌᏊ ᎠᏰᏟ 
 \newline \textcolor{red}{Sagwu ayetli}\\
Wake up! & ᎯᏰᎩ 
 \newline \textcolor{red}{hiyegi}\\
Go to sleep! & ᎯᏢᎾ 
 \newline \textcolor{red}{hitlvna}\\
days & ᏧᏒᎯᏓ 
 \newline \textcolor{red}{tsusvhida}\\
this evening & not a valid letterᎣᎯ ᏒᎯᏰᏱ 
 \newline \textcolor{red}{kohi svhiyeyi}\\
next morning & ᎤᎩᏨᏓ 
 \newline \textcolor{red}{ugitsvda}\\
watch & ᏩᏥ 
 \newline \textcolor{red}{watsi}\\
clock (lit. big watch) & ᎤᏔᎾ ᏩᏥ 
 \newline \textcolor{red}{utana watsi}\\
outside & ᏙᏱ 
 \newline \textcolor{red}{doyi}\\
\end{tabular}
\end{minipage}

\ \newline\noindent Time - Time, and Time of Day\\
JAC ago\\
morning, noon, and night\\
Ask about opening times (museum, movie)\cite{walcpp42}\cite{walcpp47}\\

\noindent In order to tell time in relationship to what has passed you would use the phrase ᏥᎨᏒ meaning "formerly existing" or "dead"\footnote{This would normally be used to refer to someone who is now dead.}\cite{feelingHiderGregg202Dpp9}\\

\begin{minipage}{\linewidth}
\begin{tabular}{p{5cm} p{9cm}}
last year & ᎡᏥ ᏥᎨᏒ 
 \newline \textcolor{red}{eji jigesv}\\
last month & ᏏᏅᏓ ᏥᎨᏒ 
 \newline \textcolor{red}{sinvda jigesv}\\
two years ago & ᏔᎵ ᎾᏕᏘᏯ 
 \newline \textcolor{red}{tali nadetiya}\\
\end{tabular}
\end{minipage}

\noindent When asked, "What time it was?" you could answer by using the phrase: ᎢᏳᏩᏂᎸᎢ - time (as it relates to striking of the clock).\cite{feelingHiderGregg202Dpp9}\\

\noindent For instance:\\
\begin{minipage}{\linewidth}
\begin{tabular}{p{5cm} p{9cm}}
It's eight o'clock. & ᏣᏁᎳ ᎢᏳᏩᏂᎸ 
 \newline \textcolor{red}{janela iyuwanilv}\\
It is five o'clock. & ᎯᏍᎩ ᎢᏳᏩᏂᎸ 
 \newline \textcolor{red}{hisgi iyuwanilv}\\
\end{tabular}
\end{minipage}

\cite{feelingHiderGregg202Dpp9}\\
\begin{minipage}{\linewidth}
\begin{tabular}{p{5cm} p{9cm}}
five before three & ᎯᏍᎩ ᎲᏓᎷᎳ ᏦᎢ 
 \newline \textcolor{red}{hisgi hvdalula joi}\\
q quarter-past & ᎩᏄᏘᏗ ᏅᎶᏒᏍᏗ 
 \newline \textcolor{red}{ginutidi nvlosvsdi}\\
it's a quarter past one. & ᎩᏄᏘᏗ ᏅᎶᏒᏍᏗ ᏌᏊᎢ 
 \newline \textcolor{red}{ginutidi nvlosvsdi sagwui}\\
It's fifteen past one. & ᏍᎩᎦᏚ ᎤᎶᏒᏍᏗ ᏌᏊ 
 \newline \textcolor{red}{sgigadu ulosvsdi sagwu}\footnote{Whether you use ᏌᏊ or ᏌᏊᎢ is your choice.  Many Cherokee drop the final -i when speaking.}\\
half past & ᎠᏰᏟ ᎤᎶᏒᏍᏗ 
 \newline \textcolor{red}{ayetli ulosvsdi}\\
it's half past nine & ᎠᏰᏟ ᎤᎶᏒᏍᏗ ᏐᏁᎳ 
 \newline \textcolor{red}{ayetli ulosvsdi sonela}\\
it's nine thirty & ᏐᏁᎳ ᏦᏍᎪᎯ 
 \newline \textcolor{red}{sonela josgohi}\\
\end{tabular}
\end{minipage}

\cite{feelingHiderGregg202Dpp9}\\
A question such as "What time is lunch?" ᎯᎳ ᎠᎠᏟᎢᎵᏒ ᎢᎦ ᎠᎵᏍᏓᏴᏗ would perhaps be answered:\"1130" ᏌᏚ ᎠᏰᏟ or "soon" ᎩᎳᏊᎢᏳᏍᏗ\footnote{Actually the phrase translates to: "kind of like later" or in English you'd say "soon"}\cite{feelingHiderGregg202Dpp9}\includepdf[pages={64}]{W:/GOOGLEDRIVE/Cherokee Umbrella/books/intermediate_cherokee_0570773_C0609_howard_gregg_eby_rick.pdf}\index{Months}
\index{}
\chapter{Months - }
\subsection{Vocabulary - ᏗᎧᏁᎢᏍᏗ 
}
\begin{minipage}{\linewidth}
\begin{tabular}{p{5cm} p{9cm}}
January & ᎤᏃᎸᏔᏂ 
 \newline \textcolor{red}{Unolvtani}\\
February & ᎧᎦᎵ 
 \newline \textcolor{red}{Kagali}\\
March & ᎠᏅᏱ 
 \newline \textcolor{red}{Anvyi}\\
April & ᎧᏬᏂ 
 \newline \textcolor{red}{Kawoni}\\
May & ᎠᎾᏍᎬᏘ 
 \newline \textcolor{red}{Anasgvti}\\
June & ᏕᎭᎷᏱ 
 \newline \textcolor{red}{Dehaluyi}\\
July & ᎫᏰᏉᏂ 
 \newline \textcolor{red}{Guyegwoni}\\
August & ᎦᎶᏂ 
 \newline \textcolor{red}{Galoni}\\
September & ᏚᎵᏍᏗ 
 \newline \textcolor{red}{Dulisdi}\\
October & ᏚᏂᏂᏗ 
 \newline \textcolor{red}{Duninidi}\\
November & ᏅᏓᏕᏆ 
 \newline \textcolor{red}{Nvdadegwa}\\
December & ᎥᏍᎩᏱ 
 \newline \textcolor{red}{Vskiyi}\\
\end{tabular}
\end{minipage}

\footnote{Discussed in the section \hyperref[sec:daysOfWeekMeaning]{Days Of Week Meanings}}\begin{minipage}{\linewidth}
\begin{tabular}{p{5cm} p{9cm}}
Today is, March, 20th. & not a valid letterᎣᎯ ᎢᎦ, ᎠᏅᏱ ᏔᎵᏍᎪᎯᏁ. 
 \newline \textcolor{red}{Kohi iga, anvyi talisgohine.}\\
Yesterday was, December fifth. & ᏒᎯ ᏥᎨᏒ, ᎥᏍᎩᏱ ᎯᏍᎩᏁ. 
 \newline \textcolor{red}{Svhi jigesv, vsgiyi hisgine.}\\
\end{tabular}
\end{minipage}

\cite{walc1pp48}\index{Seasons}
\index{}
\chapter{Seasons - }
\index{What You Will Learn}\subsection{What You Will Learn}
In this unit you will learn:
\begin{itemize}
\item JAC Seasons pp140
\item In winter
\item in summer
\item in autumn, in fall
\item in spring
\item June is in the summer
\item December is in the winter
\end{itemize}\newpage

\newpage\subsection{Vocabulary - ᏗᎧᏁᎢᏍᏗ 
}
\begin{minipage}{\linewidth}
\begin{tabular}{p{5cm} p{9cm}}
Winter & ᎪᎳ 
 \newline \textcolor{red}{Gola}\\
Spring & ᎪᎨᏱ 
 \newline \textcolor{red}{Gogeyi}\\
Summer & ᎪᎩ/ᎪᎦ 
 \newline \textcolor{red}{Gogi/Goga}\\
Autumn & ᎤᎳᎪᎲᏍᏗ 
 \newline \textcolor{red}{Ulagohvsdi}\\
\end{tabular}
\end{minipage}

\cite{walcpp49}\begin{minipage}{\linewidth}
\begin{tabular}{p{5cm} p{9cm}}
When winter comes, it becomes cold. & ᎪᎳ ᏱᏄᎵᏍᏔᎾ, Ꭰnot a valid letterᏴᏜᏗᏍᎪ’Ꭲ. 
 \newline \textcolor{red}{Gola yinulistana, ahyvdladisgo’i.}\\
The birds are singing this morning, it’s spring! & ᏥᏍᏆ ᏓᏂnot a valid letterᏃᎩ’Ꭰ not a valid letterᎣᎯ ᏌᎾᎴ, ᎪᎨᏱ’Ꮫ ᏂᎦnot a valid letterᏍᏗᎭ! 
 \newline \textcolor{red}{Jisgwa danihnogi’a kohi sanale, gogeyi’dv nigalsdiha!}\\
The leaves fall in Autumn. & ᏕᎦnot a valid letterᏍᏚᏬ’ᎣᏍᎪ’Ꭲ ᎤᎳᎪᎲᏍᏗ. 
 \newline \textcolor{red}{Degalsduwo’osgo’i ulagohvsdi.}\\
What are you going to do this summer? & ᎦᏙ ᏔᏛᏁᎵ not a valid letterᎣᎯ ᎪᎩ? 
 \newline \textcolor{red}{Gado tadvneli kohi gogi?}\\
\end{tabular}
\end{minipage}

\cite{walcpp49}\index{This and That}
\index{}
\chapter{This and That - }
\index{What You Will Learn}\subsection{What You Will Learn}
In this unit you will learn:
\begin{itemize}
\item Describing things - cars, trucks, buildings, signs
\item JAC this and that
\item this and that as describing things.  and basic here, there, this, that, where clause
\item more indepth in chapter 9
\item JAC where is it?
\item JAC here and there
\item JAC near and far
\item form phrases like this, that, these, those
\item describing things holmes smith 98
\end{itemize}\newpage

\newpage\index{To have and have not}
\index{To have and have not}
\chapter{To have and have not}
JAC do you have\\
to have and have not\\
\index{Describing Others}
\index{Describing Others}
\chapter{Describing Others}
describing people\\
\index{Directions}
\index{}
\chapter{Directions - }
\index{What You Will Learn}\subsection{What You Will Learn}
In this unit you will learn:
\begin{itemize}
\item where, where in the town center. where are my clothes
\item Giving simple directions
\item talking about more places around town and their location
\item saying what belongs to whom
\item ARC around town
\item agreement (in sentence)
\item JAC common verb forms
\item JAC asking a question
\item where is it?
\item here and there
\item near and far
\item names of places around town
\item where is the town square (center)?
\item simple directions
\item about more places in town and locations
\item what belongs to whom (continued from last chapter)
\item ch 27 pp245-252 smith holmes
\item go straight
\item make a right
\item go two blocks
\item make a left
\item second house on the right
\item go north on highway 41
\item get off at exit 12
\item make a right (head east)
\item go to the 4-way stop
\item go straight one mile
\end{itemize}\newpage

\newpage\subsection{Vocabulary - ᏗᎧᏁᎢᏍᏗ 
}
\begin{minipage}{\linewidth}
\begin{tabular}{p{5cm} p{9cm}}
North & Ꮷnot a valid letterᏴᏢᎢ 
 \newline \textcolor{red}{Juhyvdlvi}\\
South & ᏧᎦᏃᏮᎢ 
 \newline \textcolor{red}{Juganowvi}\\
Up & ᎦᎸᎳᏗ 
 \newline \textcolor{red}{Galvladi}\\
Center & ᎠᏰᏟ 
 \newline \textcolor{red}{Ayetli}\\
East & ᏗᎧᎸᎬᎢ 
 \newline \textcolor{red}{Dikalvgvi}\\
West & ᏭᏕᎵᎬᎢ 
 \newline \textcolor{red}{Wudeligvi}\\
Down & ᎡᎳᏗ 
 \newline \textcolor{red}{Eladi}\\
Towards & ᎢᏗᏜ 
 \newline \textcolor{red}{Ididla}\\
Which way? & ᎭᏢ ᎢᏗᏜ? 
 \newline \textcolor{red}{Hadlv ididla?}\\
Where?/ Where is it? & ᎭᏢ? 
 \newline \textcolor{red}{Hadlv?}\\
Where did you go? & ᎭᏢ not a valid letterᏪᏓ? 
 \newline \textcolor{red}{Hadlv hweda?}\\
Where are you heading? & ᎭᏢ Ꮻnot a valid letterᏘ? 
 \newline \textcolor{red}{Hadlv hwikti?}\\
We are heading south. & ᏧᎦᏃᏮ ᎢᏗᏜ ᏬᏥᎦᏘ. 
 \newline \textcolor{red}{Juganowv ididla wojigati.}\\
I am going to work. & ᏗᎩᎸᏫᏍᏓᏁᏗᎢ ᎨᎦ. 
 \newline \textcolor{red}{Digilvwisdanedii gega.}\\
Here & ᎠᎭᏂ 
 \newline \textcolor{red}{Ahani}\\
Close & Ꮎ’Ꭵ 
 \newline \textcolor{red}{Na’v}\\
Toward & ᎢᏗᏜ 
 \newline \textcolor{red}{Ididla}\\
Everywhere & Ꮒnot a valid letterᎥ’Ꭲ 
 \newline \textcolor{red}{Nikv’i}\\
Outside & ᏙᏱ 
 \newline \textcolor{red}{Doyi}\\
Nowhere & Ꮭ ᎢᎸᏢ 
 \newline \textcolor{red}{Tla ilvdlv}\\
\end{tabular}
\end{minipage}

\vfill\newpage\begin{minipage}{\linewidth}\begin{tabular}{p{5cm} p{9cm}}
Upstream & ᏦᎩᏗᏜ 
 \newline \textcolor{red}{Jogididla}\\
There & ᎥᎿ’Ꭲ 
 \newline \textcolor{red}{Vhna’i}\\
there & ᎠᎨ’Ꭲ ᎣᎥᎡnot a valid letter 
 \newline \textcolor{red}{Age’i Over}\\
Away from & ᎤᏟᏴ’Ꭲ 
 \newline \textcolor{red}{Udliyv’i}\\
Above & ᎦᎸᎳᏗᏜ 
 \newline \textcolor{red}{Galvladidla}\\
Below & ᎡᎳᏗᏜ 
 \newline \textcolor{red}{Eladidla}\\
Somewhere & ᎢᎸᏢ’Ꭲ 
 \newline \textcolor{red}{Ilvtlv’i}\\
Underneath & ᎭᏫᏂᏗᏜ 
 \newline \textcolor{red}{Hawinididla}\\
Downstream & Ꭸ’ᎢᏗᏜ 
 \newline \textcolor{red}{Ge’ididla}\\
Far & ᎢᎾ 
 \newline \textcolor{red}{Ina}\\
Where are you going? & ᎭᏢ ᎮᎦ? 
 \newline \textcolor{red}{Hadlv hega?}\\
When will you go? & ᎯᎳᏴ ᏖᏏ? 
 \newline \textcolor{red}{Hilayv tesi?}\\
Are you close? & Ꮎ’ᎥᏧ ᎮᏙᎭ? 
 \newline \textcolor{red}{Na’vju hedoha?}\\
Will you go there? & ᏫᏖᏙᎵᏧ? 
 \newline \textcolor{red}{Witedoliju?}\\
Where did they go? & ᎭᏢ ᎤᏁᏅᏎᎢ? 
 \newline \textcolor{red}{Hadlv unenvsei?}\\
We will go at 6:00 & ᏑᏓᎵ ᎠᏟᎢᎵᏒ ᏓᏕᏏ. 
 \newline \textcolor{red}{Sudali atliilisv dadesi.}\\
When will you be here? & ᎯᎳᏴ ᏘᎷnot a valid letterᏥ? 
 \newline \textcolor{red}{Hilayv tiluhji?}\\
He is standing over there & ᎠᎨ’Ꭲ ᏗᎦᏙᎦ. 
 \newline \textcolor{red}{Age’i digadoga.}\\
They are singing there. & ᎥᎿᎢ ᏓᏂnot a valid letterᏃᎩ’Ꭰ. 
 \newline \textcolor{red}{Vhnai danihnogi’a.}\\
I am working behind my house. & ᏗᏇᏅᏒ’Ꭲ Ꭳnot a valid letterᏂᏗᏜ ᏓᎩᎸᏫᏍᏓᏁᎭ. 
 \newline \textcolor{red}{Digwenvsv’i ohnididla dagilvwisdaneha.}\\
\end{tabular}
\end{minipage}

\vfill\newpage\begin{minipage}{\linewidth}\begin{tabular}{p{5cm} p{9cm}}
We (not you) went to the Stomp Grounds saturday night. & ᎦᏘᏲ’Ꭲ ᏬᎨᏙᎸ ᏙᏓᏈᏕᎾ ᎤᏒ’Ꭲ ᏥᎨᏒ’Ꭲ. 
 \newline \textcolor{red}{Gatiyo’i wogedolv dodagwidena usv’i jigesv’i.}\\
Him/her and I are walking towards the church. & ᏧᏂᎳᏫᏍᏗ’Ꭲ ᏗᏜ ᏬᏍᏓ’Ꭲ. 
 \newline \textcolor{red}{Junilawisdi’i didla wosda’i.}\\
They live far away. & ᎢᎾ ᏗᎨᏒ’Ꭲ ᏩᏂᏁᎳ. 
 \newline \textcolor{red}{Ina digesv’i waninela.}\\
They travel around everywhere. & Ꮒnot a valid letterᎥ’Ꭲ ᎠᏁᏙᎵᏙᎰ’Ꭲ. 
 \newline \textcolor{red}{Nikv’i anedolidoho’i.}\\
They will come here (up to this point). & ᎠᎭᏂ ᎢᏴ’Ꭲ ᏛᏂᎷnot a valid letterᏥ. 
 \newline \textcolor{red}{Ahani iyv’i dvniluhji.}\\
On top of & ᎦᏚ’Ꭲ 
 \newline \textcolor{red}{Gadu’i}\\
Over there & ᎠᎨ’Ꭲ 
 \newline \textcolor{red}{Age’i}\\
Right here & ᎠᎭᏂ 
 \newline \textcolor{red}{Ahani}\\
Chair/table & ᎦᏍᎩᎶ 
 \newline \textcolor{red}{Gasgilo}\\
In the middle/between & ᎠᏰᏟ 
 \newline \textcolor{red}{Ayetli}\\
He/she is standing & ᎦᏙᎦ 
 \newline \textcolor{red}{Gadoga}\\
I am getting up & ᎦᏗᏗ’Ꭰ 
 \newline \textcolor{red}{Gadidi’a}\\
I woke up & ᏥᏰᎩ 
 \newline \textcolor{red}{Jiyegi}\\
I am getting ready & ᎦᏛᏅ’ᎢᏍᏗᎭ 
 \newline \textcolor{red}{Gadvnv’isdiha}\\
I just took a shower & ᎦᏓᏬᏣ 
 \newline \textcolor{red}{Gadawoja}\\
I brushed my teeth & ᏕᏥnot a valid letterᏙᎬ ᏕᏥᏅᎦᎳ 
 \newline \textcolor{red}{Dejindohgv dejinvgala}\\
I am ironing & ᏕᏥᏖᏍᎦ 
 \newline \textcolor{red}{Dejitesga}\\
I went to the bathroom & ᏙᏱ-ᎡᏓᏍᏗ ᏫᏥᏴᎭ 
 \newline \textcolor{red}{Doyi-edasdi wijiyvha}\\
I just ate breakfast & ᏌᎾᎴ-ᎡᎯ ᎦᎵᏍᏓᏴᎲᎦ 
 \newline \textcolor{red}{Sanale-ehi galisdayvhvga}\\
I put on my clothes & ᏕᎦᏂᏬᏣ 
 \newline \textcolor{red}{Deganiwoja}\\
\end{tabular}
\end{minipage}

\vfill\newpage\begin{minipage}{\linewidth}\begin{tabular}{p{5cm} p{9cm}}
I just left & Ꭶ’ᏂᎩ 
 \newline \textcolor{red}{Ga’nigi}\\
I put my shoes on & ᏕᎦᎳᏑᏢᎦ 
 \newline \textcolor{red}{Degalasutlvga}\\
I combed my hair & ᎦᎵᏔᏬᏣ 
 \newline \textcolor{red}{Galitawoja}\\
I prepared my meal & Ꭰnot a valid letterᏍᏓnot a valid letterᏗ ᎦᏛᏅ’ᎢᏍᏓ 
 \newline \textcolor{red}{Alsdayhdi gadvnv’isda}\\
I just ironed my clothes & ᏗᎿᏬ ᏕᏥᏛᎳ 
 \newline \textcolor{red}{Dihnawo dejitvla}\\
\end{tabular}
\end{minipage}

\cite{walcpp828594}\noindent The verb suffix that means “Going Somewhere.”\\
Rules for use\
Attaching “-ega” to the past tense of the verb will produce the meaning of “going somewhere to do something.”\\
So take the remote past suffix and cut the “-v’i” off and add “-ega.”\\
1. Dagilvwisdanelv’i I worked. becomes —> Dagilvwisdanel-ega - I am going to work.\\
2. Agwalsdayvhnv’i I ate a meal. becomes —> Galisdayvhn-ega I am going to eat a meal.\\
3. Unasuhnv’i They fished. becomes —> Anasuhn-ega They are going to fish.\\

\noindent Place of\\
The suffixes -i and -o’i mean “the place of” and it is added to nouns in Cherokee to form place names. Examples shown below.\\
1. Agusa Creek person —>  Guso’i “Muskogee” (Creek Place)\\
2. Kolanv Raven —> Kolanv’i “Big Cove” (Place of the Raven)\\\\

On or In\\
This noun suffixes -hi and -ohi means“on” or “in” and can also have the meaning of “into” as shown below in the following examples:\\
3. Ama Water—> Amohi Into the water\\
4. Taluja Basket —>  Talujohi Into the basket\\
5. Gasgilo Table —>  Gasgilohi On the table\\
\cite{walcpp88}\index{At the doctor}
\index{}
\chapter{At the doctor - }
\index{What You Will Learn}\subsection{What You Will Learn}
In this unit you will learn:
\begin{itemize}
\item body - body parts and bodily functions
\item ARC My Head Hurts
\item ARC the future
\item ARC Command forms
\item Body parts
\item smith holmes pp236-241
\item i don't feel well
\end{itemize}\newpage

\newpage\index{To the Beach}
\index{To the Beach}
\chapter{To the Beach}
ARC let's go to the beach\\
\index{Transportation}
\index{}
\chapter{Transportation - }
\index{What You Will Learn}\subsection{What You Will Learn}
In this unit you will learn:
\begin{itemize}
\item at the bus stop
\item at the train station
\item at the hotel
\item ARC at the hotel
\item ARC nationalities
\item ARC dual form
\item JAC on the road
\item JAC bus, train, subway, taxi
\end{itemize}\newpage

\newpage\index{at the airport}
\index{at the airport}
\chapter{at the airport}
ARC at the airport\\
\index{Festivals and Celebrations}
\index{Festivals and Celebrations}
\chapter{Festivals and Celebrations}
festivals\\
\index{Money}
\index{}
\chapter{Money - }
\index{What You Will Learn}\subsection{What You Will Learn}
In this unit you will learn:
\begin{itemize}
\item ARC how much is this?
\item ARC I would like to change 100
\item buying things
\item food
\item money
\item desire
\item JAC in a restaurant
\item JAC it costs
\item JAC shopping
\item JAC house hunting
\end{itemize}\newpage

\newpage\index{Clothes and Shopping}
\index{}
\chapter{Clothes and Shopping - }
\index{What You Will Learn}\subsection{What You Will Learn}
In this unit you will learn:
\begin{itemize}
\item describe clothes
\item buy different clothing items
\end{itemize}\newpage

\newpage\index{Animals}
\index{Animals}
\chapter{Animals}
different animals in the wild\\
\index{Colors}
\index{}
\chapter{Colors - }
\index{What You Will Learn}\subsection{What You Will Learn}
In this unit you will learn:
\begin{itemize}
\item how to say dark red, dark blue, light blue, etc
\end{itemize}\newpage

\newpage\subsection{Vocabulary - ᏗᎧᏁᎢᏍᏗ 
}
\begin{minipage}{\linewidth}
\begin{tabular}{p{5cm} p{9cm}}
Black & ᎬᎿᎨᎢ 
 \newline \textcolor{red}{Gvhnagei}\\
Red & ᎩᎦᎨᎢ 
 \newline \textcolor{red}{Gigagei}\\
White & ᎤᏁᎦ 
 \newline \textcolor{red}{Unega}\\
Yellow & ᏓᎶᏂᎨᎢ 
 \newline \textcolor{red}{Dalonigei}\\
Orange & ᎠᏓᎶᏂᎨᎢ 
 \newline \textcolor{red}{Adalonigei}\\
Blue & Ꮜnot a valid letterᎣᏂᎨᎢ 
 \newline \textcolor{red}{Sakonigei}\\
Pink & ᎩᎦᎨ ᎤᏍᎪᎸ 
 \newline \textcolor{red}{Gigage Usgolv}\\
Purple & ᎠᏛᎭᎵᎨ 
 \newline \textcolor{red}{Advhalige}\\
Green & ᎢᏤ ᎢᏳᏍᏗ 
 \newline \textcolor{red}{Ije iyusdi}\\
Silver & ᎠᏕᎳ ᎤᏁᎦ 
 \newline \textcolor{red}{Adela unega}\\
Gold & ᎠᏕᎳ ᏓᎶᏂᎨ 
 \newline \textcolor{red}{Adela dalonige}\\
Brown & ᎤᏬᏗᎨ 
 \newline \textcolor{red}{Uwodige}\\
Gray/Grey & ᎤᏍᎪᎸ ᏌᎪᏂᎨ 
 \newline \textcolor{red}{usgolv sagonige}\\
Scarlet & ᎤᏍᎪᏍᏓ ᎩᎦᎨ 
 \newline \textcolor{red}{usgosda gigage}\\
tan & ᏅᏓ ᎤᎴᏴᏔᏅ 
 \newline \textcolor{red}{nvda uleyvtanv}\\
colored & ᎤᎵᏑᏫᏓ 
 \newline \textcolor{red}{ulisuwida}\\
colors & Ꮷnot a valid letterᏑᏫᏓ 
 \newline \textcolor{red}{julsuwida}\\
\end{tabular}
\end{minipage}

\index{Shapes}
\index{}
\chapter{Shapes - }
\subsection{Vocabulary - ᏗᎧᏁᎢᏍᏗ 
}
\begin{minipage}{\linewidth}
\begin{tabular}{p{5cm} p{9cm}}
Circle & ᎦᏐᏆᎸ 
 \newline \textcolor{red}{Gasogwalv}\\
Oval (long circle) & ᎦᏐᏆᎸ ᎦᏅᎯᏓ 
 \newline \textcolor{red}{Gasogwalv ganvhida}\\
Triangle (three sides) & ᏦᎢ ᏧᏅᏏᏱ 
 \newline \textcolor{red}{Joi junvsiyi}\\
Square & ᏅᎩ ᏧᏅᏏᏱ 
 \newline \textcolor{red}{Nvhgi junvsiyi}\\
Rectangle (long square) & ᎦᏅᎯᏓ ᏅᎩ ᏧᏅᏏᏱ 
 \newline \textcolor{red}{Ganvhida nvhgi junvsiyi}\\
Diamond & ᎪᏍᏓᏱ 
 \newline \textcolor{red}{Gosdayi}\\
Pentagon & ᎯᏍᎩ ᏧᏅᏏᏱ 
 \newline \textcolor{red}{Hisgi junvsiyi}\\
Star & ᏃᏈᏏ 
 \newline \textcolor{red}{Nokwisi}\\
Hexagon & ᏑᏓᎵ ᏧᏅᏏᏱ 
 \newline \textcolor{red}{Sudali junvsiyi}\\
Heptagon & ᎦᎵᏉᎩ ᏧᏅᏏᏱ 
 \newline \textcolor{red}{Galigwogi junvsiyi}\\
Octagon & ᏣᏁᎳ ᏧᏅᏏᏱ 
 \newline \textcolor{red}{Chanela junvsiyi}\\
Nonagon & ᏐᏁᎳ ᏧᏅᏏᏱ 
 \newline \textcolor{red}{Sonela junvsiyi}\\
Decagon & ᏍᎪᎯ ᏧᏅᏏᏱ 
 \newline \textcolor{red}{Sgohi junvsiyi}\\
\end{tabular}
\end{minipage}

\index{Weather}
\index{ᏙᏯᏗᏜ ᏂᎦᎵᏍᏔᏅᏍᎬ}
\chapter{Weather - ᏙᏯᏗᏜ ᏂᎦᎵᏍᏔᏅᏍᎬ}
\index{What You Will Learn}\subsection{What You Will Learn}
In this unit you will learn:
\begin{itemize}
\item JAC Weather pp29
\item How's the weather today? What's the weather like today?
\item It's nice weather.
\item It's raining.
\item It's snowing.
\item It's hot.
\item It's cold.
\item It's cool
\item It's warm
\item The weather is bad today.
\end{itemize}\newpage

\newpage\subsection{Vocabulary - ᏗᎧᏁᎢᏍᏗ 
}
\begin{minipage}{\linewidth}
\begin{tabular}{p{5cm} p{9cm}}
weather & ᏙᏯᏗᏜ ᏂᎦᎵᏍᏔᏅᏍᎬ 
 \newline \textcolor{red}{doyadidla nigalistanvsgv}\footnote{\href{https://cherokeedictionary.net/share/76306}{CED weather}}\\
weather map & ᏙᏱ ᏂᎦᎵᏍᏔᏂᏙᎲ ᎧᏃᎮᏍᎩ 
 \newline \textcolor{red}{doyi nigalistanidohv kanohesgi}\footnote{\href{https://cherokeedictionary.net/share/76304}{CED weather map}}\\
wall cloud & ᎠᏐnot a valid letterᏲᎶᎩᎳ 
 \newline \textcolor{red}{asohyologila}\footnote{\href{https://cherokeedictionary.net/share/101954}{CED wall cloud}}\\
tornado & ᎤᏃᎴ/ ᎠᎦᎷᎦ ᎤᏔᎾ 
 \newline \textcolor{red}{unole/ agaluga utana}\footnote{\href{https://cherokeedictionary.net/share/101955}{CED tornado}}\\
blizzard & ᎤᏍᎦᏎᏗ ᎫᏘᏍᎬᎢ 
 \newline \textcolor{red}{usgasedi gutisgvi}\footnote{\href{https://cherokeedictionary.net/share/101956}{CED blizzard}}\\
avalanche & ᎣᏓᎸ ᏓᏕᎵᏍᎦᎵᎲᎢ 
 \newline \textcolor{red}{odalv dadelisgalihvi}\footnote{\href{https://cherokeedictionary.net/share/101957}{CED avalanche}}\\
clear sky & ᎤᎵᎦᎵᏴᏓ 
 \newline \textcolor{red}{uligaliyvda}\footnote{\href{https://cherokeedictionary.net/share/101958}{CED clear sky}}\\
drizzle & ᎠᏍᏚᏟᏥᏙ 
 \newline \textcolor{red}{asdutlitsido}\footnote{\href{https://cherokeedictionary.net/share/101959}{CED drizzle}}\\
flood & ᏕᎦᏃᎱᎩ 
 \newline \textcolor{red}{deganohugi}\footnote{\href{https://cherokeedictionary.net/share/101960}{CED flood}}\\
snow flurries & Ꭵnot a valid letter(Ꭵ)Ꮵ  ᎦᏃᎯᎵᏙᎭ 
 \newline \textcolor{red}{vn(v)tsi  ganohilidoha}\footnote{\href{https://cherokeedictionary.net/share/101961}{CED snow flurries}}\\
freezing & ᎦᏁᏍᏓᎵᏗ 
 \newline \textcolor{red}{ganesdalidi}\footnote{\href{https://cherokeedictionary.net/share/101962}{CED freezing}}\\
freezing rain & ᏓᏲᏩᏄᎵᏓ 
 \newline \textcolor{red}{dayowanulida}\footnote{\href{https://cherokeedictionary.net/share/101963}{CED freezing rain}}\\
haze & ᏧᎦᏒᏍᏗ 
 \newline \textcolor{red}{tsugasvsdi}\footnote{\href{https://cherokeedictionary.net/share/101964}{CED haze}}\\
heavy rain & ᎦᏐᏅ'ᎥᏍᎦ / ᎦᏐᏅᎯ  ᎠᎦᏍᎦ 
 \newline \textcolor{red}{gasonv'vsga / gasonvhi  agasga}\footnote{\href{https://cherokeedictionary.net/share/101965}{CED heavy rain}}\\
high temperature (weather) & ᏩᎦᎸᎳᏗᏴ ᏄᏗnot a valid letterᎴᎬᎢ 
 \newline \textcolor{red}{wagalvladiyv nudihlegvi}\footnote{\href{https://cherokeedictionary.net/share/101966}{CED high temperature (weather)}}\\
hoarfrost & ᎤᎾᏄᏍᏗ 
 \newline \textcolor{red}{unanusdi}\footnote{\href{https://cherokeedictionary.net/share/101967}{CED hoarfrost}}\\
hurricane & ᎠᎺᏉᎯ  ᎡᏙᎲ ᎠᎦᎷᎦ 
 \newline \textcolor{red}{amequohi  edohv agaluga}\footnote{\href{https://cherokeedictionary.net/share/101968}{CED hurricane}}\\
whirlwind & ᎠᎦᎷᎦ 
 \newline \textcolor{red}{agaluga}\footnote{\href{https://cherokeedictionary.net/share/101969}{CED whirlwind}}\\
low temperature (weather) & ᏪᎳᏗᏴ ᏄᏴᏢᎢ 
 \newline \textcolor{red}{weladiyv nuyvtlvi}\footnote{\href{https://cherokeedictionary.net/share/101970}{CED low temperature (weather)}}\\
misting rain & ᎧᏅᏲᎵᏗ 
 \newline \textcolor{red}{kanvyolidi}\footnote{\href{https://cherokeedictionary.net/share/101971}{CED misting rain}}\\
\end{tabular}
\end{minipage}

\vfill\newpage\begin{minipage}{\linewidth}\begin{tabular}{p{5cm} p{9cm}}
mostly cloudy & ᎤᏟ ᎢᎦ ᎤᎶᎩᎵ 
 \newline \textcolor{red}{utli iga ulogili}\footnote{\href{https://cherokeedictionary.net/share/101972}{CED mostly cloudy}}\\
partly cloudy & ᎤᏓᏓᏟ ᎤᎶᎩᎳ 
 \newline \textcolor{red}{udadatli ulogila}\footnote{\href{https://cherokeedictionary.net/share/101973}{CED partly cloudy}}\\
rockslide & ᏅᏯ  ᎦᏒᏙᏍᎬ 
 \newline \textcolor{red}{nvya  gasvdosgv}\footnote{\href{https://cherokeedictionary.net/share/101974}{CED rockslide}}\\
mudslide & ᏝᏬᏘ  ᎬᏓᎶᏍᎬ 
 \newline \textcolor{red}{tlawoti  gvdalosgv}\footnote{\href{https://cherokeedictionary.net/share/101975}{CED mudslide}}\\
sleet & ᎦᏁᏐᎣᏍᎩ    ᎦᏁᏐ'ᎣᏍᎩ 
 \newline \textcolor{red}{ᎦᏁᏐᎣᏍᎩ    ganeso'osgi}\footnote{\href{https://cherokeedictionary.net/share/101976}{CED sleet}}\\
snow & Ꭵnot a valid letter(Ꭵ)Ꮵ 
 \newline \textcolor{red}{vn(v)tsi}\footnote{\href{https://cherokeedictionary.net/share/101977}{CED snow}}\\
snow showers & ᏗᎫᏘᏍᎩ 
 \newline \textcolor{red}{digutisgi}\footnote{\href{https://cherokeedictionary.net/share/101978}{CED snow showers}}\\
wildfire & ᏕᎦᎵᎬ 
 \newline \textcolor{red}{degaligv}\footnote{\href{https://cherokeedictionary.net/share/101979}{CED wildfire}}\\
current temperature & ᏃᏊ  ᏥᎩ ᏄᏗnot a valid letterᎴᎬᎢ 
 \newline \textcolor{red}{nogwu  tsigi nudihlegvi}\footnote{\href{https://cherokeedictionary.net/share/101980}{CED current temperature}}\\
isolated thunderstorm & ᎢᏳᏓᎵ Ꭰnot a valid letterᏴᏓᏆᎶᏍᎩ 
 \newline \textcolor{red}{iyudali ahyvdagwalosgi}\footnote{\href{https://cherokeedictionary.net/share/101981}{CED isolated thunderstorm}}\\
moderate to heavy snow & ᎠᏰᏟ  ᎠᎴ ᎦᎨᏓ  ᎫᏘᏍᎩ 
 \newline \textcolor{red}{ayehli  ale gageda  gutisgi}\footnote{\href{https://cherokeedictionary.net/share/101982}{CED moderate to heavy snow}}\\
rain and snow & ᎠᎦᏍᎩ ᎠᎴ ᎫᏘᏍᎩ 
 \newline \textcolor{red}{agasgi ale gutisgi}\footnote{\href{https://cherokeedictionary.net/share/101983}{CED rain and snow}}\\
scattered showers & ᏧᏗᎦᎴᏲᏨ ᏗᎦᏍᎩ 
 \newline \textcolor{red}{tsudigaleyotsv digasgi}\footnote{\href{https://cherokeedictionary.net/share/101984}{CED scattered showers}}\\
scattered snow showers & ᏧᏗᎦᎴᏲᏨ Ꭵnot a valid letter(Ꭵ)Ꮵ 
 \newline \textcolor{red}{tsudigaleyotsv vn(v)tsi}\footnote{\href{https://cherokeedictionary.net/share/101985}{CED scattered snow showers}}\\
scattered thunderstorms & ᏧᏗᎦᎴᏲᏨ  Ꮧnot a valid letterᏴᏓᏆᎶᏍᎩ 
 \newline \textcolor{red}{tsudigaleyotsv  dihyvdagwalosgi}\footnote{\href{https://cherokeedictionary.net/share/101986}{CED scattered thunderstorms}}\\
The wind is blowing about. & ᎦᏃᎴᎭ 
 \newline \textcolor{red}{ganoleha}\footnote{\href{https://cherokeedictionary.net/share/101987}{CED The wind is blowing about.}}\\
\end{tabular}
\end{minipage}

\index{Food}
\index{}
\chapter{Food - }
\index{What You Will Learn}\subsection{What You Will Learn}
In this unit you will learn:
\begin{itemize}
\item ARC this is delicious
\item ARC verbs
\item ARC past tense
\item JAC breakfast
\item JAC sample menu
\end{itemize}\newpage

\newpage\index{on the farm}
\index{on the farm}
\chapter{on the farm}
driving a tractor\\
harvesting corn\\
\index{Visiting Friends}
\index{Visiting Friends}
\chapter{Visiting Friends}
I am a guest\\
\index{Yours, Mine, Ours}
\index{}
\chapter{Yours, Mine, Ours - }
\index{What You Will Learn}\subsection{What You Will Learn}
In this unit you will learn:
\begin{itemize}
\item our house, your house, etc
\item JAC my, your, his/her
\item ARC to have
\item ARC possessives
\item ARC Possessive noun construction
\item JAC it's me
\item JAC it's mine
\item JAC about me
\item JAC to me
\end{itemize}\newpage

\newpage\index{Questions}
\index{Questions}
\chapter{Questions}
JAC who, what, when, where, how\\
JAC how much\\
JAC how many\\
ARC question words\\
\index{Family}
\index{}
\chapter{Family - }
\index{What You Will Learn}\subsection{What You Will Learn}
In this unit you will learn:
\begin{itemize}
\item talking about your family, saying who things belong to, describing things
\item numbers 21-100
\item ARC the family
\item JAC family members
\item ADD more family members
\item Split clans to their own book section
\end{itemize}\newpage

\newpage\subsection{Vocabulary - ᏗᎧᏁᎢᏍᏗ 
}
\begin{minipage}{\linewidth}
\begin{tabular}{p{5cm} p{9cm}}
Family & ᏏᏓᏁᎸ’Ꭲ 
 \newline \textcolor{red}{Sidanelv’i}\\
Clans & ᏧᏂᏴᏫ 
 \newline \textcolor{red}{Juniyvwi}\\
Father & ᎡᏙᏓ 
 \newline \textcolor{red}{Edoda}\\
Mother & ᎡᏥ 
 \newline \textcolor{red}{Eji}\\
(paternal) Aunt & ᎡᏠᎩ 
 \newline \textcolor{red}{Etlogi}\\
Uncle & ᎡᏚᏥ 
 \newline \textcolor{red}{Eduji}\\
Grandma & ᎡᎵᏏ 
 \newline \textcolor{red}{Elisi}\\
Grandpa (maternal) & ᎡᏚᏓ 
 \newline \textcolor{red}{Eduda}\\
My sibling (opposite sex) & ᎥᎩᏙ 
 \newline \textcolor{red}{Vgido}\\
My brother (same sex) & ᏦᏍᏓᏓnot a valid letterᏅᏟ 
 \newline \textcolor{red}{Josdadahnvtli}\\
My sister (same sex) & ᏦᏍᏓᏓᎸ’Ꭲ 
 \newline \textcolor{red}{Josdadalv’i}\\
My cousin & ᎪᎱᏍᏗ ᎠᏋᏂ 
 \newline \textcolor{red}{Gohusdi agwvni}\\
My nephew & ᎠᏆᏓᏚᏥᏴ ᎠᏧᏣ 
 \newline \textcolor{red}{Agwadadujiyv achuja}\\
My niece & ᎠᏆᏓᏚᏥᏴ ᎠᎨnot a valid letterᏳᏣ 
 \newline \textcolor{red}{Agwadadujiyv agehyuja}\\
Aunt (maternal) & Ꮤ’ᎵᏁ ᎡᏥ 
 \newline \textcolor{red}{Ta’line eji}\\
Grandpa (paternal) & ᎡᏂᏏ 
 \newline \textcolor{red}{Enisi}\\
Deer Clan & ᎠᏂ-ᎧᏫ 
 \newline \textcolor{red}{Ani-Kawi}\\
Savannah Clan & ᎠᏂ-ᎪᏓᎨᏫ 
 \newline \textcolor{red}{Ani-godagewi}\\
Bird Clan & ᎠᏂ-ᏥᏍᏆ 
 \newline \textcolor{red}{Ani-jisgwa}\\
Blue Clan & ᎠᏂ-ᏌᎰᏂ 
 \newline \textcolor{red}{Ani-sahoni}\\
\end{tabular}
\end{minipage}

\vfill\newpage\begin{minipage}{\linewidth}\begin{tabular}{p{5cm} p{9cm}}
Wolf Clan & ᎠᏂ-Ꮹnot a valid letter(Ꭰ)Ꮿ 
 \newline \textcolor{red}{Ani-wah(a)ya}\\
Paint Clan & ᎠᏂ-ᏬᏗ 
 \newline \textcolor{red}{Ani-wodi}\\
Long Hair Clan & ᎠᏂ-ᎩᎶᎯ 
 \newline \textcolor{red}{Ani-gilohi}\\
My clan & ᏗᎩᏴᏫ 
 \newline \textcolor{red}{Digiyvwi}\\
My clan is \underline{    } &  ᏗᎩᏴᏫ. 
 \newline \textcolor{red}{ digiyvwi.}\\
Your clan is \underline{    }. &  ᏗᏣᏴᏫ. 
 \newline \textcolor{red}{ dijayvwi.}\\
\end{tabular}
\end{minipage}

\index{People and Professions}
\index{}
\chapter{People and Professions - }
\index{What You Will Learn}\subsection{What You Will Learn}
In this unit you will learn:
\begin{itemize}
\item JAC profession
\item Personal details
\item What do you do?
\item Different jobs and occupations
\item personal details [not family unless "I have 2 kids"]
\item At work - looking for a job
\item at the office
\item ARC when do you work
\item pictures of professions
\end{itemize}\newpage

\newpage\subsection{Vocabulary - ᏗᎧᏁᎢᏍᏗ 
}
\begin{minipage}{\linewidth}
\begin{tabular}{p{5cm} p{9cm}}
Person & ᏴᏫ 
 \newline \textcolor{red}{Yvwi}\\
Lawyer & ᏗᏘᏲᎯ 
 \newline \textcolor{red}{Ditiyohi}\\
Treasurer & ᎠᏕᎳ ᎠᎦᏘᏯ 
 \newline \textcolor{red}{Adela agatiya}\\
People & ᎠᏂᏴᏫ 
 \newline \textcolor{red}{Aniyvwi}\\
Preacher & ᎠᎵᏣᏙᎲᏍᎩ 
 \newline \textcolor{red}{Alijadohvsgi}\\
Translator & ᏗᏁᏢᏗᏍᎩ 
 \newline \textcolor{red}{Dinetlvdisgi}\\
Teacher & ᏗᏕᏲᎲᏍᎩ 
 \newline \textcolor{red}{Dideyohvsgi}\\
Dancer & ᎠᎵᏍᎩᏍᎩ 
 \newline \textcolor{red}{Alisgisgi}\\
Artist & ᏗᏟᎶᏍᏔᏅᎲᏍᎩ 
 \newline \textcolor{red}{Ditlilostanvhvsgi}\\
Doctor & ᎦᎾᎦᏘ 
 \newline \textcolor{red}{Ganagati}\\
Medicine person & ᏗᏓᏅᏫᏍᎩ 
 \newline \textcolor{red}{Didanvwisgi}\\
Worker/Employee & ᏧᎸᏫᏍᏓᏁ 
 \newline \textcolor{red}{Julvwisdane}\\
Nurse & ᏧᏂᏢᎩ ᏗᎦᏘᏯ 
 \newline \textcolor{red}{Junidlvgi digatiya}\\
Chief / President & ᎤᎬᏫᏳᎯ 
 \newline \textcolor{red}{Ugvwiyuhi}\\
Accountant & ᎠᏕᎳ ᎤᎦᏎᏍᏗ 
 \newline \textcolor{red}{Adela Ugasesdi}\\
Actor & ᎠᏛᏁᎵᏍᎩ 
 \newline \textcolor{red}{Advnelisgi}\\
Policeman & ᏗᏓᏂᏱᏍᎩ 
 \newline \textcolor{red}{Didaniyisgi}\\
Firefighter & ᏗᎦᏜᏗᏍᎩ 
 \newline \textcolor{red}{Digadladisgi}\\
Singer & ᏗᎧᏃᎩᏍᎩ 
 \newline \textcolor{red}{Dikanogisgi}\\
Soldier & ᎠᏲᏍᎩ 
 \newline \textcolor{red}{Ayosgi}\footnote{"Ayawisgi" is an alternative and older way to write the word.}\\
\end{tabular}
\end{minipage}

\vfill\newpage\begin{minipage}{\linewidth}\begin{tabular}{p{5cm} p{9cm}}
Store clerk & ᎠᏓᎾᏅ ᎠᏓᎦᏘᏕᎯ 
 \newline \textcolor{red}{Adananv adagatidehi}\\
Student & ᏗᏕᎶᏆᏍᎩ 
 \newline \textcolor{red}{Didelogwasgi}\\
Writer & ᏗᎪᏪᎵᏍᎩ 
 \newline \textcolor{red}{Digowelisgi}\\
Boss & ᏧᎧᏍᏟ /ᎠᏍᎦᏰᎬᏍᏗ 
 \newline \textcolor{red}{Jukastli /Asgayegvsdi}\\
\end{tabular}
\end{minipage}

\
\index{Attaching Pronoun Prefixes To Nouns}\subsection{Attaching Pronoun Prefixes To Nouns}
In Cherokee, pronoun prefixes are attached to a noun just as they are in a verb. Pronouns are not separated from the noun and verb as they are in English. The following examples will begin to demonstrate how to attach set A and set B prefixes onto nouns.\cite{walcpp53}\\
\newline \noindent Set A Prefixes:\begin{itemize}
\item A-ganakti —> He/she is a doctor
\item Ji-ganakti —> I am a doctor
\item Ani-ganakti —> They are doctors
\item Hi-ganakti —-> You are a doctor
\item A-yvwi —> A person or He/she is a person
\item Ji-yvwi —> I am a person
\item Hi-yvwi —> You are a person
\item Ani-yvwi —> They are people
\end{itemize}

\noindent Set B Prefixes:
\begin{itemize}
\item U-gvwiyuhi —> He/she is a chief/president
\item Agi-gvwiyuhi —> I am a chief/president
\item Ja-gvwiyuhi —> You are a chief/president
\item Uni-gvwiyuhi —> They are chiefs/presidents
\end{itemize}
\index{Sort Further}
\index{Sort Further}
\chapter{Sort Further}
JAC nouns and noun particles\\
JAC common adjective forms\\
JAC plain or polite\\
JAC to construct polite\\
JAC some comparisons\\
JAC also\\
JAC I have been to...\\
JAC sometimes I go\\
JAC i can.  I am able to.\\
JAC I've decided to\\
JAC the modifiers\\
JAC the noun maker no\\
JAC to goJAC a few action phrases\\
JAC reading section\\
JAC they say that\\
JAC I have to, I must\\
JAC something to drink\\
JAC a little and a little\\
JAC too much\\
JAC more or less\\
JAC enough and some more\\
JAC I want to\\
JAC i intend to\\
JAC it is supposed to\\
JAC something, everything, nothing\\
JAC of course, it's a pitty it doesn' tmatter\\
JAC the same\\
JAC already\\
JAC i like it, it's good\\
JAC i don't like it, it's bad\\
JAC reading\\
JAC some, someone, something\\
JAC once, twice\\
JAC up to\\
JAC i need. it is necessary\\
JAC i feel like\\
JAC at the home of\\
JAC in, on, under\\
JAC if, when\\
JAC without\\
JAC to come\\
JAC to say\\
JAC to do\\
JAC i'm a stranger here\\
Present, past, future\\
a story or part of a story in Cherokee - then analyze it\\
negative command sentences\\
\subsection{Vocabulary - ᏗᎧᏁᎢᏍᏗ 
}
\begin{minipage}{\linewidth}
\begin{tabular}{p{5cm} p{9cm}}
How much does it cost?  What are the stakes? & ᎭᎳ ᏧᎬᏩᎶᏗ 
 \newline \textcolor{red}{hala jugvwalodi}\\
What size do you want?  (What written do you want?) & ᎭᎳ ᎪᏪᎵ ᏣᏚᎵᎭ? 
 \newline \textcolor{red}{Hala goweli jaduliha?}\\
How many do you want? How much do you want? & ᎭᎳ ᎢᎦᎢ ᏣᏚᎵᎭ? 
 \newline \textcolor{red}{Hala igai jaduliha?}\\
David quit his job. & ᏓᏫᏗ ᎤᏑᎳᎪᏤ ᏚᎸᏫᏍᏔᏁᎲ. 
 \newline \textcolor{red}{dawidi usulagoje dulvwistanehv.}\\
That's the way it is! & ᏍᎩᏛ ᏄᏍᏗ 
 \newline \textcolor{red}{sgidv nusdi}\\
(Yes), and several workers in his plant have lost their jobs. & ᎠᎴ ᎯᎸᏍᎩ ᏧᏂᎸᏫᏍᏔᏁᎯ ᏚᏂᏲᎱᏏ ᏧᏂᎸᏍᏔᏁᏘ. 
 \newline \textcolor{red}{Ale hilvsgi junilvwistanehi duniyohusi junilvstaneti.}\\
What day of the week is it today? & ᎦᏙ ᎤᏍᏗ ᎢᎦ ᎠᏟᎩᎵ? 
 \newline \textcolor{red}{Gado usdi iga atligili?}\\
What date is it today? & ᎦᏙ ᎤᏍᏗ ᎠᏎᏍᏗ not a valid letterᎣᎯ? 
 \newline \textcolor{red}{Gado usdi asesdi kohi?}\\
I am glad you all came. & ᎦᎵᎡᎵᎩ ᏂᎦᏓ ᏥᏥᎷᎩ. 
 \newline \textcolor{red}{Galieligi nigada jijilugi.}\\
Won't you sit down?  (anywhere you choose) To one person: (bend yourself on something) To several: (Let's all be sitting around) & ᎭᎵᏍᏚᏢᎦ. ᎢᏓᏅᏂᏓ. 
 \newline \textcolor{red}{Halisdutlvga. idanvnida.}\\
Everybody be seated. (in a certain place like at the dinner table) & ᎯᎦᏓ ᎢᏣᏅᏂᏓ. 
 \newline \textcolor{red}{higada ijanvnida.}\\
rain (always a verb) & ᎠᎦᏍᎦ 
 \newline \textcolor{red}{agasga}\\
lawyer (arguer for a goal) & ᎩᏂᏘᏲᎯᎯ 
 \newline \textcolor{red}{ginitiyohihi}\\
marine (on-deep-water-goer) & ᎠᎹᏱᎭᏫᏂᎠᏁᏙᎯ 
 \newline \textcolor{red}{amayihawinianedohi}\\
minister & ᎠᎾᎵᏣᏙᎲᏍᎩ 
 \newline \textcolor{red}{analijadohvsgi}\\
musician & ᏗᏂᎧᏃᎩᏍᏗᏍᎩ 
 \newline \textcolor{red}{dinikanogisdisgi}\\
nurse (caretaker of the ill) & ᏧᎾᏢᎩ ᏗᏂᎩᏘᏯ 
 \newline \textcolor{red}{junatlvgi dinigitiya}\\
post & ᏗᎪᏪᎵᏍᎩ ᎧᏃᎭᏢᏍᎩ 
 \newline \textcolor{red}{digowelisgi kanohatlvsgi}\\
posts & ᏗᏃᏪᎵᏍᎩ ᎧᏃᎮᏢᏍᎩ 
 \newline \textcolor{red}{dinowelisgi kanohetlvsgi}\\
policeman (final catcher) & ᏗᏂᏓᏂᏱᏍᎩ 
 \newline \textcolor{red}{dinidaniyisgi}\\
\end{tabular}
\end{minipage}

\vfill\newpage\begin{minipage}{\linewidth}\begin{tabular}{p{5cm} p{9cm}}
repairman (makes it good again) & ᎣᏍᏓ ᎢᎬᏁᎯ 
 \newline \textcolor{red}{osda igvnehi}\\
repairman & ᎣᏍᏓ ᎢᏯᏅᏁᎯ 
 \newline \textcolor{red}{osda iyanvnehi}\\
sailor (on-water-goer) & ᎠᎹᎢᎠᏁᏙᎯ 
 \newline \textcolor{red}{amaianedohi}\\
secretary (writer-down) & ᎪᏪᎵᏍᎩ ᏗᎪᏪᎵᏍᎩ 
 \newline \textcolor{red}{gowelisgi digowelisgi}\\
secretaries & ᏗᎪᏪᎵᏍᎩ ᏗᏃᏪᎵᏍᎩ 
 \newline \textcolor{red}{digowelisgi dinowelisgi}\\
soldier & ᎠᏂᏲᏍᎩ 
 \newline \textcolor{red}{aniyosgi}\\
teacher & ᏗᎾᏕᏲᎲᏍᎩ 
 \newline \textcolor{red}{dinadeyohvsgi}\\
typist & ᏗᏂ ᎨᏍᏗᏍᎩ 
 \newline \textcolor{red}{dini gesdisgi}\\
When do they get off work? (each day) & ᎭᎳᏴ ᎠᏂᏑᎳᎪᎪ ᏧᏂᎸᏫᏍᏔᏁᎯ? 
 \newline \textcolor{red}{Halayv anisulagogo junilvwistanehi?}\\
When do you stop work? (temporarily, for time off or for vacation) & ᎭᎳᏴ Ꮩnot a valid letterᏖᏙᎵ? 
 \newline \textcolor{red}{Halayv tohtedoli?}\\
John has a new job. & ᏣᏂ ᎠᏤ ᏧᎸᏫᏍᏓᏂᏘ ᎤᎭ. 
 \newline \textcolor{red}{Jani aje julvwisdaniti uha.}\\
I hear he is working very hard. & ᎦᏛᎩ ᏍᏔᏲᏒ ᏚᎸᏫᏍᏔᏁᎲ 
 \newline \textcolor{red}{Gadvgi stayosv dulvwistanehv}\\
They say Mark has been transferred. & ᎤᏣᏘᎾ ᏚᎸᏫᏍᏔᏁ ᎠᎾᏗ ᎹᎦ. 
 \newline \textcolor{red}{ujatina dulvwistane anadi Maga.}\\
\end{tabular}
\end{minipage}

\index{NOTES:}
\index{ᏓᏓᏚᎬ ᎪᏪᎵ 
}
\chapter{NOTES: - ᏓᏓᏚᎬ ᎪᏪᎵ 
}
\
\index{Dialect Breakdown}\subsection{Dialect Breakdown - ᎣᏔᎵ  ᎩᏚᏩ}
The Giduwah, or Eastern, dialect of Cherokee varies in some ways from the Otali, or Western, dialect dialect of Cherokee.  A simple example is ᎭᏩ (G) vs ᎰᏩ (O).  Different spellings, same word.  Both mean "ok, alright, sure".  The word "ᎰᏩ" is an affirmative response and can be understood to mean different things depending on how it is used. Two of the more common meanings are "Okay" and "You are welcome".\cite{joynerlesson4}

\label{sec:wordBreakdownTohiOsi}\section{Word Breakdown - ᏙᎯ and ᎣᏏ Tohi and Osi}Altman and Belt (pp91-92) have this to say about Tohi and Osi:Tohi is a Cherokee morpheme that indicates the state in which nature is flowing at its appropriate pace and everything is as it should be. This fundamental concept is used in greetings and responses (\textcolor{red}{Tohigwatsv?} and \textcolor{red}{Tohigwu.}), and in a variety of other instances and constructions that indicate an underlying concern with the notion that things be flowing well in the Cherokee world. Tohi can be glossed variously as "well," "peaceful," "unhurried," and "health." In the Cherokee speakers' view, if the state of tohi becomes disrupted there can be disastrous consequences, and communities that are disrupted in this way can be dangerous or unhealthy places to live.\\
\\
In addition to and as an adjunct to tohi, the concept of osi describes the proper state of the individual person. Visualized as upright, facing forward, and resting on a single point of balance, osi is also used in greetings and replies (\textcolor{red}{osigwatsv?} and \textcolor{red}{osigwu.}), and in other contexts that indicate that the notion of an individual’s state of being is crucial in ensuring that all is flowing well in the larger Cherokee world. Osi is properly understood as referring to the state of neutrality and balance, but it is most often glossed as "good." If individuals are out of balance, they can cause problems in the larger system.\cite{altmanBelt90-98}

\label{sec:daysOfWeekMeaning}\section{Word Breakdown - Notes on the meanings of the days of the week}Notes on the meanings of the days of the week:\\
\cite{walc1pp46}\\

\textit{Unadodagwonvi} - When they have completed doing something all day\\
\textit{Ta’line iga} - The second day\\
\textit{Jo’ine iga} - The third day\\
\textit{Nvhgine iga} - The fourth day\\
\textit{Jun(v)gilosdi} - The day they wash their clothes\\
\footnote{The first way to say Friday was actually "hisgine'iga" which means "the fifth day."}\textit{Unadodagwidena} - The day before they do something all day (when you went to town)\\
\textit{Unadodagwasgv’i} - The day they do something all day.\\
\index{JAC public notices and signs}
\index{JAC public notices and signs}
\chapter{JAC public notices and signs}
\index{JAC summary of japanese grammar}
\index{JAC summary of japanese grammar}
\chapter{JAC summary of japanese grammar}
\end{multicols}