\index{Pronunciation and Syllabary}
\index{}
\chapter{Pronunciation and Syllabary - }
\includepdf[pages={9-10, 109}]{/projects/GoogleDriveTimo/Cherokee Umbrella/sort/timo/walc1.pdf}
\index{Greet Others}
\index{ᎤᎾᎵᎮᎵᏤᏘ}
\chapter{Greet Others - ᎤᎾᎵᎮᎵᏤᏘ}
\index{What You Will Learn}\subsection{What You Will Learn}
In this unit you will learn:
\begin{itemize}
\item How to greet people
\item How to make short descriptive phrases
\item About definites and indefinites
\item Say "hello" and "goodbye"
\item Ask how someone is doing
\end{itemize}\newpage

\newpage\subsection{Dialog - ᎠᎾᎵᏃᎮᏍᎬ}
\begin{tabular}{p{2cm} p{11cm}}
ᏓᏂᎵ:\newline \textcolor{red}{Danili}: & ᎣᏏᏲ.  ᏓᏂᎵ ᏓᏩᏙᎠ.  ᎦᏙ ᏕᏣᏙᎠ? 
\newline\textcolor{red}{Osiyo.  Danili dawadoa.  Gado dejadoa?}\\
ᏑᏌᏂ:\newline \textcolor{red}{Susani}: & ᎣᏏᏲ.  ᏑᏌᏂ ᏓᏩᏙᎠ.  ᏙᎯᏧ? 
\newline\textcolor{red}{Osiyo.  Susani dawadoa.  Dohiju?}\\
ᏓᏂᎵ:\newline \textcolor{red}{Danili}: & ᎣᏏᏊ.  ᏂᎯᎾᎲ? 
\newline\textcolor{red}{Osigwu.  Nihinahv?}\\
ᏑᏌᏂ:\newline \textcolor{red}{Susani}: & ᎣᏍᏓ.  ᏙᎾᏓᎪᎲᎢ. 
\newline\textcolor{red}{Osda.  Donadagohvi.}\\
ᏓᏂᎵ:\newline \textcolor{red}{Danili}: & ᏙᎾᏓᎪᎲᎢ. 
\newline\textcolor{red}{Donadagohvi.}\\
\end{tabular}
\\
\\
\\
\noindent\begin{tabular}{p{2cm} p{11cm}}Daniel: & Hello.  My name is Daniel.  What is your name?\\
Susan: & Hello.  My name is Susan.  How are you?\\
Daniel: & I'm fine.  And you?\\
Susan: & Good.  Goodbye.\\
Daniel: & Goodbye.\\
\end{tabular}
\vfill\newpage\subsection{Vocabulary - ᏗᎧᏁᎢᏍᏗ 
}
\begin{minipage}{\linewidth}
\begin{tabular}{p{3cm} p{11cm}}
Titus & ᏓᏓᏏ 
 \newline \textcolor{red}{Dadasi}\\
Timothy & ᏗᎹᏗ 
 \newline \textcolor{red}{Dimadi}\\
Mary & ᎺᎵ 
 \newline \textcolor{red}{Meli}\\
Mark & ᎹᎦ 
 \newline \textcolor{red}{Maga}\\
Daniel & ᏓᏂᎵ 
 \newline \textcolor{red}{Danili}\\
Susan & ᏑᏌᏂ 
 \newline \textcolor{red}{Susani}\\
John & ᏣᏂ 
 \newline \textcolor{red}{Jani}\\
Hello & (Ꭳ)ᏏᏲ 
 \newline \textcolor{red}{(o)siyo}\\
And & ᎠᎴ 
  or ᏃᎴ 
 \newline \textcolor{red}{ale <e>or nole}\\
Good & ᎣᏍᏓ 
 \newline \textcolor{red}{osda}\\
Well/Fine & ᎣᏏᏊ 
 \newline \textcolor{red}{osigwu}\\
\end{tabular}
\end{minipage}

\index{Hello}\subsection{Hello - ᎣᏏᏲ}
In Cherokee there is only one way to say 'Good Morning,' 'Good Afternoon,' 'Good Evening,' and Hello; that is by saying ᎣᏏᏲ \textcolor{red}{(o)siyo} or the shortened version ᏏᏲ \textcolor{red}{siyo}.\footnote{We will follow the convention of placing optional syllables in parenthesis.  You will see this written as (O)siyo.  The parentheses around the 'O' mean that the voicing of the 'O' is optional.}\footnote{Osi will be discussed more in the section \hyperref[sec:wordBreakdownTohiOsi]{Word Breakdown - Tohi and Osi}}
\section{Exercise - ᎠᎵᏏᎾᎯᏍᏗᏍᎬ ᏗᎬᏙᏗ}
Translate to Cherokee syllabary and the phonetic equivalent\\
1. Hello, Mary 2. Hello, Mark 3. Hello, Daniel 4. Hello, Susan\\
1. (O)siyo, Meli. 2. (O)siyo, Maga. 3. (O)siyo, Danili 4. (O)siyo, Susani\\
\index{Goodbye}\subsection{Goodbye - ᏙᎾᏓᎪᎲᎢ}
There is no word for 'goodbye' only 'to meet again'. The way to say 'goodbye' to one person is ᏙᎾᏓᎪᎲᎢ \textcolor{red}{donadagohvi}. If you would like to say 'goodbye' to more than one person you would say ᏙᏓᏓᎪᎲᎢ \textcolor{red}{dodadagohvi}.  Lit: Let's meet again.\footnote{We will discuss the plurality prefixes (d-) in the section \hyperref[sec:wordBreakdownPluralityPrefixes]{Word Breakdown - Plurality Prefixes}}
\section{Exercise - ᎠᎵᏏᎾᎯᏍᏗᏍᎬ ᏗᎬᏙᏗ}
Translate to Cherokee syllabary and the phonetic equivalent\\
1. Goodbye, Mary and John. 2. Goodbye, Titus. 3. Goodbye, Daniel. 4. Goodbye, Mary, John, Susan, and Mark.\\
1. Dodadagohvi, Meli ale Jani. 2. Donadagohvi, Dadasi 3. Donadagohvi, Danili 4. Dodadagohvi, Meli, Jani, Susani, ale Maga\\
\index{What is your name?}
\index{ᎦᏙ ᏕᏣᏙᎠ?}
\chapter{What is your name? - ᎦᏙ ᏕᏣᏙᎠ?}
\index{What You Will Learn}\subsection{What You Will Learn}
In this unit you will learn:
\begin{itemize}
\item REMOVE
\end{itemize}\newpage

\newpage\subsection{Dialog - ᎠᎾᎵᏃᎮᏍᎬ}
\begin{tabular}{p{2cm} p{11cm}}
not a valid letterᎡᎼᎥᎡ:\newline \textcolor{red}{REMOVE}: & not a valid letterᎡᎼᎥᎡ 
\newline\textcolor{red}{REMOVE}\\
not a valid letterᎡᎼᎥᎡ:\newline \textcolor{red}{REMOVE}: & not a valid letterᎡᎼᎥᎡ 
\newline\textcolor{red}{REMOVE}\\
not a valid letterᎡᎼᎥᎡ:\newline \textcolor{red}{REMOVE}: & not a valid letterᎡᎼᎥᎡ 
\newline\textcolor{red}{REMOVE}\\
not a valid letterᎡᎼᎥᎡ:\newline \textcolor{red}{REMOVE}: & not a valid letterᎡᎼᎥᎡ 
\newline\textcolor{red}{REMOVE}\\
not a valid letterᎡᎼᎥᎡ:\newline \textcolor{red}{REMOVE}: & not a valid letterᎡᎼᎥᎡ 
\newline\textcolor{red}{REMOVE}\\
not a valid letterᎡᎼᎥᎡ:\newline \textcolor{red}{REMOVE}: & not a valid letterᎡᎼᎥᎡ 
\newline\textcolor{red}{REMOVE}\\
not a valid letterᎡᎼᎥᎡ:\newline \textcolor{red}{REMOVE}: & not a valid letterᎡᎼᎥᎡ 
\newline\textcolor{red}{REMOVE}\\
not a valid letterᎡᎼᎥᎡ:\newline \textcolor{red}{REMOVE}: & not a valid letterᎡᎼᎥᎡ 
\newline\textcolor{red}{REMOVE}\\
not a valid letterᎡᎼᎥᎡ:\newline \textcolor{red}{REMOVE}: & not a valid letterᎡᎼᎥᎡ 
\newline\textcolor{red}{REMOVE}\\
not a valid letterᎡᎼᎥᎡ:\newline \textcolor{red}{REMOVE}: & not a valid letterᎡᎼᎥᎡ 
\newline\textcolor{red}{REMOVE}\\
\end{tabular}
\\
\\
\\
\noindent\begin{tabular}{p{2cm} p{11cm}}Mary: & Hello.  How are you?  My name is Mary.  What is your name?\\
Daniel: & I am fine.  My name is Daniel.  This is my friend.  His name is John.\\
Mary: & Hello.  My name is Mary.  What is your name?\\
Daniel: & Hi.  My name is daniel.  How are you?\\
Mary: & I am fine.  And you?\\
Daniel: & I am fine too.  Is everything ok with you? (Is everything fine?)\\
Mary: & Yes everything is fine.  I am happy that I have seen you.\\
Daniel: & I am happy to have seen you too.\\
Mary: & Who is that?\\
Daniel: & That is John.  His name is John.\\
\end{tabular}
\vfill\newpage\subsection{Vocabulary - ᏗᎧᏁᎢᏍᏗ 
}
\begin{minipage}{\linewidth}
\begin{tabular}{p{3cm} p{11cm}}
REMMOVE & not a valid letterᎡᎼᎥᎡ 
 \newline \textcolor{red}{REMOVE}\\
\end{tabular}
\end{minipage}


    Meeting people pp 2-3 (name, to want)

    Simple questions pp4 (tsu, sgo, sg, s) and pp 74 Smith

    turn these statements into questions
    turn these questions into statements
    ask for xyz
    do you want xyz
    tell your friend you would like an apple
    Identify other people you see that you are not currently talking to.

    More on this in Chapter 4

    Is your name bob?
    Is his name Barry?
    \index{I'd like you to meet...}
\index{I'd like you to meet...}
\chapter{I'd like you to meet...}
[could be in What is your name?]\\
JAC I'd like you to meet\\
JAC What's new\\
JAC have you two met?\\
JAC glad to have met you\\
\index{I want, I see}
\index{I want, I see}
\chapter{I want, I see}
Deal with plurals and I we/want from Ch4\\
I see 2 apples\\
I want 4 bananas\\
I want some grapes\\
there are some verbs included in the book specifically for this purpose\\
\index{Where are you from?}
\index{Where are you from?}
\chapter{Where are you from?}
ARC where are you from?\\
tell about oklahoma and nc\\
I am from xyz\\
pics of nationalities\\
reference Appendicies\\
ex10 where are these people from?\\
ex 12 list xyz of countries (area codes of states?)\\
where are you?  where are you from?\\
languages - german\\
countries - Germany\\
people - German\\
english - yonega (according-to-white)\\
\index{Do you speak Tsalagi?}
\index{Do you speak Tsalagi?}
\chapter{Do you speak Tsalagi?}
Do you speak xyz?\\
JAC do you speak Japanese?\\
JAC please speak a little slower\\
other languages\\
\index{on the telephone}
\index{on the telephone}
\chapter{on the telephone}
ARC the telephone number is\\
JAC telephoning\\
what is your phone number?\\
ex11 what are the phone numbers and who do they belong to?\\
contacts list on your phone\\
email list?\\
\index{Address \& Email}
\index{Address \& Email}
\chapter{Address \& Email}
JAC my address is\\
JAC writing and mailing letters\\
email - NOT JAC - but faxing didn't seem like a current topic\\
\index{Numbers}
\index{ᏗᏎᏍᏗ}
\chapter{Numbers - ᏗᏎᏍᏗ}
\index{What You Will Learn}\subsection{What You Will Learn}
In this unit you will learn:
\begin{itemize}
\item Tell time
\end{itemize}\newpage

\newpage\subsection{Dialog - ᎠᎾᎵᏃᎮᏍᎬ}
\begin{tabular}{p{2cm} p{11cm}}
not a valid letterᎡᎼᎥᎡ:\newline \textcolor{red}{REMOVE}: & not a valid letterᎡᎼᎥᎡ 
\newline\textcolor{red}{REMOVE}\\
\end{tabular}
\\
\\
\\
\noindent\begin{tabular}{p{2cm} p{11cm}}REMOVE: & REMOVE\\
\end{tabular}
\vfill\newpage\subsection{Vocabulary - ᏗᎧᏁᎢᏍᏗ 
}
\begin{minipage}{\linewidth}
\begin{tabular}{p{3cm} p{11cm}}
one & ᏌᏊ 
 \newline \textcolor{red}{saquu}\\
two & ᏔᎵ 
 \newline \textcolor{red}{tali}\\
three & ᏦᎢ 
 \newline \textcolor{red}{tsoi}\\
four & ᏅᎯᎩ 
 \newline \textcolor{red}{nvhigi}\\
five & ᎯᏍᎩ 
 \newline \textcolor{red}{hisgi}\\
six & ᏑᏓᎵ 
 \newline \textcolor{red}{sudali}\\
seven & ᎦᎵᏉᎩ 
 \newline \textcolor{red}{galiquogi}\\
eight & ᏣᏁᎳ 
 \newline \textcolor{red}{tsanela}\\
nine & ᏐᏁᎳ 
 \newline \textcolor{red}{sonela}\\
ten & ᏍᎪᎯ 
 \newline \textcolor{red}{sgohi}\\
first & ᎢᎬᏱᎢ 
 \newline \textcolor{red}{igvyii}\\
second & ᏔᎵᏁᎢ 
 \newline \textcolor{red}{talinei}\\
third & ᏦᎢᏁᎢ 
 \newline \textcolor{red}{tsoinei}\\
fourth & ᏅᏍᎩᏁᎢ 
 \newline \textcolor{red}{nvsginei}\\
fifth & ᎯᏍᎩᏁᎢ 
 \newline \textcolor{red}{hisginei}\\
sixth & ᏑᏓᎵᏁᎢ 
 \newline \textcolor{red}{sudalinei}\\
seventh & ᎦᎵᏉᎩᏁᎢ 
 \newline \textcolor{red}{galiquoginei}\\
eighth & ᏧᏁᎵᏁᎢ 
 \newline \textcolor{red}{tsunelinei}\\
ninth & ᏐᏁᎵᏁᎢ 
 \newline \textcolor{red}{sonelinei}\\
tenth & ᏍᎪᎯᏁᎢ 
 \newline \textcolor{red}{sgohinei}\\
\end{tabular}
\end{minipage}

\index{Cardinal Numbers}\subsection{Cardinal Numbers}
Cardinal Numbers are any of the numbers that express amount, as one, two, three,  etc. (distinguished from ordinal number).\cite{cardinalNumbers}Cardinal numbers answer the question: How many are there? and tell the total.\index{Ordinal Numbers}\subsection{Ordinal Numbers}
Cardinal numbers are any of the numbers that express degree, quality, or position in a series, as first, second, and third  (distinguished from cardinal number ).\cite{ordinalNumbers}Ordinal numbers answer the question: Where does it fit in a numbered set? and tell the order.Ord(inal) - Ord(er)\index{Dates}
\index{}
\chapter{Dates - }
\index{What You Will Learn}\subsection{What You Will Learn}
In this unit you will learn:
\begin{itemize}
\item REMOVE
\end{itemize}\newpage

\newpage\subsection{Dialog - ᎠᎾᎵᏃᎮᏍᎬ}
\begin{tabular}{p{2cm} p{11cm}}
not a valid letterᎡᎼᎥᎡ:\newline \textcolor{red}{REMOVE}: & not a valid letterᎡᎼᎥᎡ 
\newline\textcolor{red}{REMOVE}\\
\end{tabular}
\\
\\
\\
\noindent\begin{tabular}{p{2cm} p{11cm}}REMOVE: & REMOVE\\
\end{tabular}
\vfill\newpage\subsection{Vocabulary - ᏗᎧᏁᎢᏍᏗ 
}
\begin{minipage}{\linewidth}
\begin{tabular}{p{3cm} p{11cm}}
Monday & ᎤᎾᏙᏓᏉᏅᎢ 
 \newline \textcolor{red}{Unadodagwonvi}\\
Tuesday & ᏔᎵᏁ ᎢᎦ 
 \newline \textcolor{red}{Taline iga}\\
Wednesday & ᏦᎢᏁ ᎢᎦ 
 \newline \textcolor{red}{Joine iga}\\
Thursday & ᏅᎩᏁ ᎢᎦ 
 \newline \textcolor{red}{Nvhgine iga}\\
Friday & ᏧᏅᎩᎶᏍᏗ 
 \newline \textcolor{red}{Junvgilosdi}\\
Saturday & ᎤᎾᏙᏓᏈᏕᎾ 
 \newline \textcolor{red}{Unadodagwidena}\\
Sunday & ᎤᎾᏙᏓᏆᏍᎬᎢ 
 \newline \textcolor{red}{Unadodagwasgvi}\\
\end{tabular}
\end{minipage}

\footnote{Discussed in the section \hyperref[sec:daysOfWeekMeaning]{Days Of Week Meanings}}\index{Time, Counting, Hours, Minutes, Seconds, Fractions}
\index{}
\chapter{Time, Counting, Hours, Minutes, Seconds, Fractions - }
\index{What You Will Learn}\subsection{What You Will Learn}
In this unit you will learn:
\begin{itemize}
\item REMOVE
\end{itemize}\newpage

\newpage\subsection{Dialog - ᎠᎾᎵᏃᎮᏍᎬ}
\begin{tabular}{p{2cm} p{11cm}}
ᏓᏂᎵ:\newline \textcolor{red}{Danili}: & ᎯᎳ ᎠᏟᎢᎵᏒ? 
\newline\textcolor{red}{Hila atliilisv?}\\
\end{tabular}
\\
\\
\\
\noindent\begin{tabular}{p{2cm} p{11cm}}Daniel: & At what time?\\
\end{tabular}
\vfill\newpage\subsection{Vocabulary - ᏗᎧᏁᎢᏍᏗ 
}
\begin{minipage}{\linewidth}
\begin{tabular}{p{3cm} p{11cm}}
Hours & ᎢᏧᏟᎶᏓ 
 \newline \textcolor{red}{Ijutliloda}\\
Minutes & ᎢᏯᏔᏬᏍᏔᏅ 
 \newline \textcolor{red}{Iyatawostanv}\\
Seconds & ᎢᏯᏎᏢ 
 \newline \textcolor{red}{Iyasetlv}\\
Before & ᎤᏓᎷᎳ 
 \newline \textcolor{red}{Udalula}\\
After & ᎤᎶᏒᏍᏗ 
 \newline \textcolor{red}{Ulosvsdi}\\
Today & ᎪᎯ ᎢᎦ 
 \newline \textcolor{red}{Gohi iga}\\
Tomorrow & ᏌᎾᎴ ᎢᏴ 
 \newline \textcolor{red}{Sanale iyv}\\
Morning & ᏌᎾᎴ ᏗᏜ 
 \newline \textcolor{red}{Sanale didla}\\
Yesterday & ᏒᎯ 
 \newline \textcolor{red}{Svhi}\\
Dawn & ᎤᎩᏥᏕᏱ 
 \newline \textcolor{red}{Ugitsideyi}\\
Afternoon & ᏒᎯᏰᏱ ᏗᏜ 
 \newline \textcolor{red}{Svhiyeyi didla}\footnote{Any time after 12:00 p.m. until the sun starts to set.}\\
Evening & ᏒᎯᏰᏱ 
 \newline \textcolor{red}{Svhiyeyi}\footnote{The time of day when the sun is setting.}\\
Night & ᎤᏒ 
 \newline \textcolor{red}{Usv}\\
Midnight & ᏒᏃᏱ 
 \newline \textcolor{red}{Svnoyi}\footnote{The time somewhere in the late time of night like 12:00 a.m.}\\
Day/ Noon & ᎢᎦ 
 \newline \textcolor{red}{Iga}\\
At what time? & ᎯᎳ ᎠᏟᎢᎵᏒ? 
 \newline \textcolor{red}{Hila atliilisv?}\\
What time is it?\\ (what hour is it) & ᎯᎳ ᎢᏳᏩᏂᎸ? 
 \newline \textcolor{red}{Hila iyuwanilv?}\\
What time is it? & ᎯᎳ ᎠᏟᎢᎵ? 
 \newline \textcolor{red}{Hila atliili?}\\
When are you going? & ᎯᎳᏴ ᏖᏏ? 
 \newline \textcolor{red}{Hilayv tesi?}\\
It is 8:00 & ᏣᏁᎳ ᎢᏳᏩᏂᎸ. 
 \newline \textcolor{red}{Chanela iyuwanilv.}\\
\end{tabular}
\end{minipage}

\vfill\newpage\begin{minipage}{\linewidth}\begin{tabular}{p{3cm} p{11cm}}
1 Hour & ᏑᏟᎶᏓ 
 \newline \textcolor{red}{Sutliloda}\\
When will it start? & ᎯᎳᏴ ᏛᏓᎴᏅᎯ? 
 \newline \textcolor{red}{Hilayv dvdalenvhi?}\\
When will it end? & ᎯᎳᏴ ᏛᎵᏍᏆᏗ? 
 \newline \textcolor{red}{Hilayv dvlisgwadi?}\\
9:45 (fifteen minutes before ten) & ᏍᎩᎦᏚ ᎢᏯᏔᏬᏍᏔᏅ ᎤᏓᎷᎳ ᏍᎪᎯ 
 \newline \textcolor{red}{Sgigadu iyatawostanv udalula sgohi}\\
10:15 (fifteen minutes after ten) & ᏍᎩᎦᏚ ᎢᏯᏔᏬᏍᏔᏅ ᎤᎶᏒᏍᏗ ᏍᎪᎯ 
 \newline \textcolor{red}{Sgigadu iyatawostanv ulosvsdi sgohi}\\
1:30 (one and a half) & ᏌᏊ ᎠᏰᏟ 
 \newline \textcolor{red}{Sagwu ayetli}\\
Wake up! & ᎯᏰᎩ 
 \newline \textcolor{red}{hiyegi}\\
Go to sleep! & ᎯᏢᎾ 
 \newline \textcolor{red}{hitlvna}\\
days & ᏧᏒᎯᏓ 
 \newline \textcolor{red}{tsusvhida}\\
this evening & not a valid letterᎣᎯ ᏒᎯᏰᏱ 
 \newline \textcolor{red}{kohi svhiyeyi}\\
next morning & ᎤᎩᏨᏓ 
 \newline \textcolor{red}{ugitsvda}\\
watch & ᏩᏥ 
 \newline \textcolor{red}{watsi}\\
clock (lit. big watch) & ᎤᏔᎾ ᏩᏥ 
 \newline \textcolor{red}{utana watsi}\\
\end{tabular}
\end{minipage}

Time - Time, and Time of Day\\
JAC ago\\
morning, noon, and night\\
Ask about opening times (museum, movie)\\
\cite{walcpp42}\cite{walcpp47}\index{Months}
\index{}
\chapter{Months - }
\index{What You Will Learn}\subsection{What You Will Learn}
In this unit you will learn:
\begin{itemize}
\item REMOVE
\end{itemize}\newpage

\newpage\subsection{Dialog - ᎠᎾᎵᏃᎮᏍᎬ}
\begin{tabular}{p{2cm} p{11cm}}
not a valid letterᎡᎼᎥᎡ:\newline \textcolor{red}{REMOVE}: & not a valid letterᎡᎼᎥᎡ 
\newline\textcolor{red}{REMOVE}\\
\end{tabular}
\\
\\
\\
\noindent\begin{tabular}{p{2cm} p{11cm}}REMOVE: & REMOVE\\
\end{tabular}
\vfill\newpage\subsection{Vocabulary - ᏗᎧᏁᎢᏍᏗ 
}
\begin{minipage}{\linewidth}
\begin{tabular}{p{3cm} p{11cm}}
January & ᎤᏃᎸᏔᏂ 
 \newline \textcolor{red}{Unolvtani}\\
February & ᎧᎦᎵ 
 \newline \textcolor{red}{Kagali}\\
March & ᎠᏅᏱ 
 \newline \textcolor{red}{Anvyi}\\
April & ᎧᏬᏂ 
 \newline \textcolor{red}{Kawoni}\\
May & ᎠᎾᏍᎬᏘ 
 \newline \textcolor{red}{Anasgvti}\\
June & ᏕᎭᎷᏱ 
 \newline \textcolor{red}{Dehaluyi}\\
July & ᎫᏰᏉᏂ 
 \newline \textcolor{red}{Guyegwoni}\\
August & ᎦᎶᏂ 
 \newline \textcolor{red}{Galoni}\\
September & ᏚᎵᏍᏗ 
 \newline \textcolor{red}{Dulisdi}\\
October & ᏚᏂᏂᏗ 
 \newline \textcolor{red}{Duninidi}\\
November & ᏅᏓᏕᏆ 
 \newline \textcolor{red}{Nvdadegwa}\\
December & ᎥᏍᎩᏱ 
 \newline \textcolor{red}{Vskiyi}\\
\end{tabular}
\end{minipage}

\footnote{Discussed in the section \hyperref[sec:daysOfWeekMeaning]{Days Of Week Meanings}}\index{Seasons}
\index{}
\chapter{Seasons - }
\index{What You Will Learn}\subsection{What You Will Learn}
In this unit you will learn:
\begin{itemize}
\item REMOVE
\end{itemize}\newpage

\newpage\subsection{Dialog - ᎠᎾᎵᏃᎮᏍᎬ}
\begin{tabular}{p{2cm} p{11cm}}
not a valid letterᎡᎼᎥᎡ:\newline \textcolor{red}{REMOVE}: & not a valid letterᎡᎼᎥᎡ 
\newline\textcolor{red}{REMOVE}\\
\end{tabular}
\\
\\
\\
\noindent\begin{tabular}{p{2cm} p{11cm}}REMOVE: & REMOVE\\
\end{tabular}
\vfill\newpage\subsection{Vocabulary - ᏗᎧᏁᎢᏍᏗ 
}
\begin{minipage}{\linewidth}
\begin{tabular}{p{3cm} p{11cm}}
Winter & ᎪᎳ 
 \newline \textcolor{red}{Gola}\\
Spring & ᎪᎨᏱ 
 \newline \textcolor{red}{Gogeyi}\\
Summer & ᎪᎩ/ᎪᎦ 
 \newline \textcolor{red}{Gogi/Goga}\\
Autumn & ᎤᎳᎪᎲᏍᏗ 
 \newline \textcolor{red}{Ulagohvsdi}\\
\end{tabular}
\end{minipage}

\cite{walcpp49}\noindent JAC Seasons pp140\\
In winter\\
in summer\\
in autumn, in fall\\
in spring\\
June is in the summer\\
December is in the winter\\
\index{Profession}
\index{Profession}
\chapter{Profession}
JAC profession\\
Personal details\\
What do you do?\\
Different jobs and occupations\\
personal details [not family unless "I have 2 kids"]\\
At work - looking for a job\\
at the office\\
ARC when do you work\\
pictures of profesions\\
\index{This and That}
\index{This and That}
\chapter{This and That}
Describing things - cars, trucks, buildings, signs\\
JAC this and that\\
this and that as describing things.  and basic here, there, this, that, where clause\\
more indepth in chapter 9\\
JAC where is it?\\
JAC here and there\\
JAC near and far\\
form phrases like this, that, these, those\\
describing things holmes smith 98\\
\index{Family}
\index{Family}
\chapter{Family}
talking about your family, saying who things belong to, describing things\\
numbers 21-100\\
ARC the family\\
JAC family members\\
\index{To have and have not}
\index{To have and have not}
\chapter{To have and have not}
JAC do you have\\
to have and have not\\
\index{Describing Others}
\index{Describing Others}
\chapter{Describing Others}
describing people\\
\index{Directions}
\index{Directions}
\chapter{Directions}
where, where in the town center. where are my clothes\\
Giving simple directions\\
talking about more places around town and their location\\
saying what belongs to whom\\
ARC around town\\
agreement (in sentence)\\
JAC common verb forms\\
JAC asking a question\\
where is it?\\
here and there\\
near and far\\
names of places around town\\
where is the town square (center)?\\
simple directions\\
about more places in town and locations\\
what belongs to whom (continued from last chapter)\\
ch 27 pp245-252 smith holmes\\
go straight\\
make a right\\
go two blocks\\
make a left\\
second house on the right\\
go north on highway 41\\
get off at exit 12\\
make a right (head east)\\
go to the 4-way stop\\
go straight one mile\\
\index{At the doctor}
\index{At the doctor}
\chapter{At the doctor}
body - body parts and bodily functions\\
ARC My Head Hurts\\
ARC the future\\
ARC Command forms\\
Body parts\\
smith holmes pp236-241\\
i don't feel well\\
\index{To the Beach}
\index{To the Beach}
\chapter{To the Beach}
ARC let's go to the beach\\
\index{Transportation}
\index{Transportation}
\chapter{Transportation}
at the bus stop\\
at the train station\\
at the hotel\\
ARC at the hotel\\
ARC nationalities\\
ARC dual form\\
JAC on the road\\
JAC bus, train, subway, taxi\\
\index{at the airport}
\index{at the airport}
\chapter{at the airport}
ARC at the airport\\
\index{Festivals and Celebrations}
\index{Festivals and Celebrations}
\chapter{Festivals and Celebrations}
festivals\\
\index{Weather}
\index{Weather}
\chapter{Weather}
JAC How is the weather\\
\index{Money}
\index{Money}
\chapter{Money}
ARC how much is this?\\
ARC I would like to change 100\\
buying things\\
food\\
money\\
desire\\
JAC in a restaurant\\
JAC it costs\\
JAC shopping\\
JAC house hunting\\
\index{Clothes and Shopping}
\index{Clothes and Shopping}
\chapter{Clothes and Shopping}
describe clothes\\
buy different clothing items\\
\index{Animals}
\index{Animals}
\chapter{Animals}
different animals in the wild\\
\index{Colors}
\index{}
\chapter{Colors - }
\index{What You Will Learn}\subsection{What You Will Learn}
In this unit you will learn:
\begin{itemize}
\item REMOVE
\end{itemize}\newpage

\newpage\subsection{Dialog - ᎠᎾᎵᏃᎮᏍᎬ}
\begin{tabular}{p{2cm} p{11cm}}
not a valid letterᎡᎼᎥᎡ:\newline \textcolor{red}{REMOVE}: & not a valid letterᎡᎼᎥᎡ 
\newline\textcolor{red}{REMOVE}\\
\end{tabular}
\\
\\
\\
\noindent\begin{tabular}{p{2cm} p{11cm}}REMOVE: & REMOVE\\
\end{tabular}
\vfill\newpage\subsection{Vocabulary - ᏗᎧᏁᎢᏍᏗ 
}
\begin{minipage}{\linewidth}
\begin{tabular}{p{3cm} p{11cm}}
REMMOVE & not a valid letterᎡᎼᎥᎡ 
 \newline \textcolor{red}{REMOVE}\\
\end{tabular}
\end{minipage}

\index{Shapes}
\index{}
\chapter{Shapes - }
\index{What You Will Learn}\subsection{What You Will Learn}
In this unit you will learn:
\begin{itemize}
\item REMOVE
\end{itemize}\newpage

\newpage\subsection{Dialog - ᎠᎾᎵᏃᎮᏍᎬ}
\begin{tabular}{p{2cm} p{11cm}}
not a valid letterᎡᎼᎥᎡ:\newline \textcolor{red}{REMOVE}: & not a valid letterᎡᎼᎥᎡ 
\newline\textcolor{red}{REMOVE}\\
\end{tabular}
\\
\\
\\
\noindent\begin{tabular}{p{2cm} p{11cm}}REMOVE: & REMOVE\\
\end{tabular}
\vfill\newpage\subsection{Vocabulary - ᏗᎧᏁᎢᏍᏗ 
}
\begin{minipage}{\linewidth}
\begin{tabular}{p{3cm} p{11cm}}
REMMOVE & not a valid letterᎡᎼᎥᎡ 
 \newline \textcolor{red}{REMOVE}\\
Circle & ᎦᏐᏆᎸ 
 \newline \textcolor{red}{Gasogwalv}\\
Oval (long circle) & ᎦᏐᏆᎸ ᎦᏅᎯᏓ 
 \newline \textcolor{red}{Gasogwalv ganvhida}\\
Triangle (three sides) & ᏦᎢ ᏧᏅᏏᏱ 
 \newline \textcolor{red}{Joi junvsiyi}\\
Square & ᏅᎩ ᏧᏅᏏᏱ 
 \newline \textcolor{red}{Nvhgi junvsiyi}\\
Rectangle (long square) & ᎦᏅᎯᏓ ᏅᎩ ᏧᏅᏏᏱ 
 \newline \textcolor{red}{Ganvhida nvhgi junvsiyi}\\
Diamond & ᎪᏍᏓᏱ 
 \newline \textcolor{red}{Gosdayi}\\
Pentagon & ᎯᏍᎩ ᏧᏅᏏᏱ 
 \newline \textcolor{red}{Hisgi junvsiyi}\\
Star & ᏃᏈᏏ 
 \newline \textcolor{red}{Nokwisi}\\
Hexagon & ᏑᏓᎵ ᏧᏅᏏᏱ 
 \newline \textcolor{red}{Sudali junvsiyi}\\
Heptagon & ᎦᎵᏉᎩ ᏧᏅᏏᏱ 
 \newline \textcolor{red}{Galigwogi junvsiyi}\\
Octagon & ᏣᏁᎳ ᏧᏅᏏᏱ 
 \newline \textcolor{red}{Chanela junvsiyi}\\
Nonagon & ᏐᏁᎳ ᏧᏅᏏᏱ 
 \newline \textcolor{red}{Sonela junvsiyi}\\
Decagon & ᏍᎪᎯ ᏧᏅᏏᏱ 
 \newline \textcolor{red}{Sgohi junvsiyi}\\
\end{tabular}
\end{minipage}

\index{Weather}
\index{ᏙᏯᏗᏜ ᏂᎦᎵᏍᏔᏅᏍᎬ}
\chapter{Weather - ᏙᏯᏗᏜ ᏂᎦᎵᏍᏔᏅᏍᎬ}
\index{What You Will Learn}\subsection{What You Will Learn}
In this unit you will learn:
\begin{itemize}
\item REMOVE
\end{itemize}\newpage

\newpage\subsection{Dialog - ᎠᎾᎵᏃᎮᏍᎬ}
\begin{tabular}{p{2cm} p{11cm}}
not a valid letterᎡᎼᎥᎡ:\newline \textcolor{red}{REMOVE}: & not a valid letterᎡᎼᎥᎡ 
\newline\textcolor{red}{REMOVE}\\
\end{tabular}
\\
\\
\\
\noindent\begin{tabular}{p{2cm} p{11cm}}REMOVE: & REMOVE\\
\end{tabular}
\vfill\newpage\subsection{Vocabulary - ᏗᎧᏁᎢᏍᏗ 
}
\begin{minipage}{\linewidth}
\begin{tabular}{p{3cm} p{11cm}}
weather & ᏙᏯᏗᏜ ᏂᎦᎵᏍᏔᏅᏍᎬ 
 \newline \textcolor{red}{doyadidla nigalistanvsgv}\footnote{\href{https://cherokeedictionary.net/share/76306}{CED weather}}\\
weather map & ᏙᏱ ᏂᎦᎵᏍᏔᏂᏙᎲ ᎧᏃᎮᏍᎩ 
 \newline \textcolor{red}{doyi nigalistanidohv kanohesgi}\footnote{\href{https://cherokeedictionary.net/share/76304}{CED weather map}}\\
wall cloud & ᎠᏐnot a valid letterᏲᎶᎩᎳ 
 \newline \textcolor{red}{asohyologila}\footnote{\href{https://cherokeedictionary.net/share/101954}{CED wall cloud}}\\
tornado & ᎤᏃᎴ/ ᎠᎦᎷᎦ ᎤᏔᎾ 
 \newline \textcolor{red}{unole/ agaluga utana}\footnote{\href{https://cherokeedictionary.net/share/101955}{CED tornado}}\\
blizzard & ᎤᏍᎦᏎᏗ ᎫᏘᏍᎬᎢ 
 \newline \textcolor{red}{usgasedi gutisgvi}\footnote{\href{https://cherokeedictionary.net/share/101956}{CED blizzard}}\\
avalanche & ᎣᏓᎸ ᏓᏕᎵᏍᎦᎵᎲᎢ 
 \newline \textcolor{red}{odalv dadelisgalihvi}\footnote{\href{https://cherokeedictionary.net/share/101957}{CED avalanche}}\\
clear sky & ᎤᎵᎦᎵᏴᏓ 
 \newline \textcolor{red}{uligaliyvda}\footnote{\href{https://cherokeedictionary.net/share/101958}{CED clear sky}}\\
drizzle & ᎠᏍᏚᏟᏥᏙ 
 \newline \textcolor{red}{asdutlitsido}\footnote{\href{https://cherokeedictionary.net/share/101959}{CED drizzle}}\\
flood & ᏕᎦᏃᎱᎩ 
 \newline \textcolor{red}{deganohugi}\footnote{\href{https://cherokeedictionary.net/share/101960}{CED flood}}\\
snow flurries & Ꭵnot a valid letter(Ꭵ)Ꮵ  ᎦᏃᎯᎵᏙᎭ 
 \newline \textcolor{red}{vn(v)tsi  ganohilidoha}\footnote{\href{https://cherokeedictionary.net/share/101961}{CED snow flurries}}\\
freezing & ᎦᏁᏍᏓᎵᏗ 
 \newline \textcolor{red}{ganesdalidi}\footnote{\href{https://cherokeedictionary.net/share/101962}{CED freezing}}\\
freezing rain & ᏓᏲᏩᏄᎵᏓ 
 \newline \textcolor{red}{dayowanulida}\footnote{\href{https://cherokeedictionary.net/share/101963}{CED freezing rain}}\\
haze & ᏧᎦᏒᏍᏗ 
 \newline \textcolor{red}{tsugasvsdi}\footnote{\href{https://cherokeedictionary.net/share/101964}{CED haze}}\\
heavy rain & ᎦᏐᏅ'ᎥᏍᎦ / ᎦᏐᏅᎯ  ᎠᎦᏍᎦ 
 \newline \textcolor{red}{gasonv'vsga / gasonvhi  agasga}\footnote{\href{https://cherokeedictionary.net/share/101965}{CED heavy rain}}\\
high temperature (weather) & ᏩᎦᎸᎳᏗᏴ ᏄᏗnot a valid letterᎴᎬᎢ 
 \newline \textcolor{red}{wagalvladiyv nudihlegvi}\footnote{\href{https://cherokeedictionary.net/share/101966}{CED high temperature (weather)}}\\
hoarfrost & ᎤᎾᏄᏍᏗ 
 \newline \textcolor{red}{unanusdi}\footnote{\href{https://cherokeedictionary.net/share/101967}{CED hoarfrost}}\\
hurricane & ᎠᎺᏉᎯ  ᎡᏙᎲ ᎠᎦᎷᎦ 
 \newline \textcolor{red}{amequohi  edohv agaluga}\footnote{\href{https://cherokeedictionary.net/share/101968}{CED hurricane}}\\
whirlwind & ᎠᎦᎷᎦ 
 \newline \textcolor{red}{agaluga}\footnote{\href{https://cherokeedictionary.net/share/101969}{CED whirlwind}}\\
low temperature (weather) & ᏪᎳᏗᏴ ᏄᏴᏢᎢ 
 \newline \textcolor{red}{weladiyv nuyvtlvi}\footnote{\href{https://cherokeedictionary.net/share/101970}{CED low temperature (weather)}}\\
misting rain & ᎧᏅᏲᎵᏗ 
 \newline \textcolor{red}{kanvyolidi}\footnote{\href{https://cherokeedictionary.net/share/101971}{CED misting rain}}\\
\end{tabular}
\end{minipage}

\vfill\newpage\begin{minipage}{\linewidth}\begin{tabular}{p{3cm} p{11cm}}
mostly cloudy & ᎤᏟ ᎢᎦ ᎤᎶᎩᎵ 
 \newline \textcolor{red}{utli iga ulogili}\footnote{\href{https://cherokeedictionary.net/share/101972}{CED mostly cloudy}}\\
partly cloudy & ᎤᏓᏓᏟ ᎤᎶᎩᎳ 
 \newline \textcolor{red}{udadatli ulogila}\footnote{\href{https://cherokeedictionary.net/share/101973}{CED partly cloudy}}\\
rockslide & ᏅᏯ  ᎦᏒᏙᏍᎬ 
 \newline \textcolor{red}{nvya  gasvdosgv}\footnote{\href{https://cherokeedictionary.net/share/101974}{CED rockslide}}\\
mudslide & ᏝᏬᏘ  ᎬᏓᎶᏍᎬ 
 \newline \textcolor{red}{tlawoti  gvdalosgv}\footnote{\href{https://cherokeedictionary.net/share/101975}{CED mudslide}}\\
sleet & ꭶꮑꮠꭳꮝꭹ    ᎦᏁᏐ'ᎣᏍᎩ 
 \newline \textcolor{red}{ᎦᏁᏐᎣᏍᎩ    ganeso'osgi}\footnote{\href{https://cherokeedictionary.net/share/101976}{CED sleet}}\\
snow & Ꭵnot a valid letter(Ꭵ)Ꮵ 
 \newline \textcolor{red}{vn(v)tsi}\footnote{\href{https://cherokeedictionary.net/share/101977}{CED snow}}\\
snow showers & ᏗᎫᏘᏍᎩ 
 \newline \textcolor{red}{digutisgi}\footnote{\href{https://cherokeedictionary.net/share/101978}{CED snow showers}}\\
wildfire & ᏕᎦᎵᎬ 
 \newline \textcolor{red}{degaligv}\footnote{\href{https://cherokeedictionary.net/share/101979}{CED wildfire}}\\
current temperature & ᏃᏊ  ᏥᎩ ᏄᏗnot a valid letterᎴᎬᎢ 
 \newline \textcolor{red}{nogwu  tsigi nudihlegvi}\footnote{\href{https://cherokeedictionary.net/share/101980}{CED current temperature}}\\
isolated thunderstorm & ᎢᏳᏓᎵ Ꭰnot a valid letterᏴᏓᏆᎶᏍᎩ 
 \newline \textcolor{red}{iyudali ahyvdagwalosgi}\footnote{\href{https://cherokeedictionary.net/share/101981}{CED isolated thunderstorm}}\\
moderate to heavy snow & ᎠᏰᏟ  ᎠᎴ ᎦᎨᏓ  ᎫᏘᏍᎩ 
 \newline \textcolor{red}{ayehli  ale gageda  gutisgi}\footnote{\href{https://cherokeedictionary.net/share/101982}{CED moderate to heavy snow}}\\
rain and snow & ᎠᎦᏍᎩ ᎠᎴ ᎫᏘᏍᎩ 
 \newline \textcolor{red}{agasgi ale gutisgi}\footnote{\href{https://cherokeedictionary.net/share/101983}{CED rain and snow}}\\
scattered showers & ᏧᏗᎦᎴᏲᏨ ᏗᎦᏍᎩ 
 \newline \textcolor{red}{tsudigaleyotsv digasgi}\footnote{\href{https://cherokeedictionary.net/share/101984}{CED scattered showers}}\\
scattered snow showers & ᏧᏗᎦᎴᏲᏨ Ꭵnot a valid letter(Ꭵ)Ꮵ 
 \newline \textcolor{red}{tsudigaleyotsv vn(v)tsi}\footnote{\href{https://cherokeedictionary.net/share/101985}{CED scattered snow showers}}\\
scattered thunderstorms & ᏧᏗᎦᎴᏲᏨ  Ꮧnot a valid letterᏴᏓᏆᎶᏍᎩ 
 \newline \textcolor{red}{tsudigaleyotsv  dihyvdagwalosgi}\footnote{\href{https://cherokeedictionary.net/share/101986}{CED scattered thunderstorms}}\\
The wind is blowing about. & ᎦᏃᎴᎭ 
 \newline \textcolor{red}{ganoleha}\footnote{\href{https://cherokeedictionary.net/share/101987}{CED The wind is blowing about.}}\\
\end{tabular}
\end{minipage}

JAC Weather pp29\\
How's the weather today? What's the weather like today?\\
It's nice weather.\\
It's raining.\\
It's snowing.\\
It's hot.\\
It's cold.\\
It's cool\\
It's warm\\
The weather is bad today.\\
\index{Food}
\index{Food}
\chapter{Food}
ARC this is delicious\\
ARC verbs\\
ARC past tense\\
JAC breakfast\\
JAC sample menu\\
\index{on the farm}
\index{on the farm}
\chapter{on the farm}
driving a tractor\\
harvesting corn\\
\index{Visiting Friends}
\index{Visiting Friends}
\chapter{Visiting Friends}
I am a guest\\
\index{Yours, Mine, Ours}
\index{Yours, Mine, Ours}
\chapter{Yours, Mine, Ours}
our house, your house, etc\\
JAC my, your, his/her\\
ARC to have\\
ARC possessives\\
ARC Possessive noun construction\\
JAC it's me\\
JAC it's mine\\
JAC about me\\
JAC to me\\
\index{Questions}
\index{Questions}
\chapter{Questions}
JAC who, what, when, where, how\\
JAC how much\\
JAC how many\\
ARC question words\\
\index{Sort Further}
\index{Sort Further}
\chapter{Sort Further}
JAC nouns and noun particles\\
JAC common adjective forms\\
JAC plain or polite\\
JAC to construct polite\\
JAC some comparisons\\
JAC also\\
JAC I have been to...\\
JAC sometimes I go\\
JAC i can.  I am able to.\\
JAC I've decided to\\
JAC the modifiers\\
JAC the noun maker no\\
JAC to goJAC a few action phrases\\
JAC reading section\\
JAC they say that\\
JAC I have to, I must\\
JAC something to drink\\
JAC a little and a little\\
JAC too much\\
JAC more or less\\
JAC enough and some more\\
JAC I want to\\
JAC i intend to\\
JAC it is supposed to\\
JAC something, everything, nothing\\
JAC of course, it's a pitty it doesn' tmatter\\
JAC the same\\
JAC already\\
JAC i like it, it's good\\
JAC i don't like it, it's bad\\
JAC reading\\
JAC some, someone, something\\
JAC once, twice\\
JAC up to\\
JAC i need. it is necessary\\
JAC i feel like\\
JAC at the home of\\
JAC in, on, under\\
JAC if, when\\
JAC without\\
JAC to come\\
JAC to say\\
JAC to do\\
JAC i'm a stranger here\\
Present, past, future\\
a story or part of a story in Cherokee - then analyze it\\
negative command sentences\\
\index{NOTES:}
\index{ᏓᏓᏚᎬ ᎪᏪᎵ 
}
\chapter{NOTES: - ᏓᏓᏚᎬ ᎪᏪᎵ 
}
\index{Dialect Breakdown}\subsection{Dialect Breakdown - ᎣᏔᎵ  ᎩᏚᏩ}
The Giduwah, or Eastern, dialect of Cherokee varies in some ways from the Otali, or Western, dialect dialect of Cherokee.  A simple example is ᎭᏩ (G) vs ᎰᏩ (O).  Different spellings, same word.  Both mean "ok, alright, sure".  The word "ᎰᏩ" is an affirmative response and can be understood to mean different things depending on how it is used. Two of the more common meanings are "Okay" and "You are welcome".\cite{joynerlesson4}

\label{sec:wordBreakdownTohiOsi}\section{Word Breakdown - ᏙᎯ and ᎣᏏ Tohi and Osi}Altman and Belt (pp91-92) have this to say about Tohi and Osi:Tohi is a Cherokee morpheme that indicates the state in which nature is flowing at its appropriate pace and everything is as it should be. This fundamental concept is used in greetings and responses (\textcolor{red}{Tohigwatsv?} and \textcolor{red}{Tohigwu.}), and in a variety of other instances and constructions that indicate an underlying concern with the notion that things be flowing well in the Cherokee world. Tohi can be glossed variously as "well," "peaceful," "unhurried," and "health." In the Cherokee speakers' view, if the state of tohi becomes disrupted there can be disastrous consequences, and communities that are disrupted in this way can be dangerous or unhealthy places to live.In addition to and as an adjunct to tohi, the concept of osi describes the proper state of the individual person. Visualized as upright, facing forward, and resting on a single point of balance, osi is also used in greetings and replies (\textcolor{red}{osigwatsv?} and \textcolor{red}{osigwu.}), and in other contexts that indicate that the notion of an individual’s state of being is crucial in ensuring that all is flowing well in the larger Cherokee world. Osi is properly understood as referring to the state of neutrality and balance, but it is most often glossed as "good." If individuals are out of balance, they can cause problems in the larger system.\cite{altmanBelt90-98}

\label{sec:daysOfWeekMeaning}\section{Word Breakdown - Notes on the meanings of the days of the week}Notes on the meanings of the days of the week:\\
\cite{walc1pp46}\textit{Unadodagwonvi} - When they have completed doing something all day\\
\textit{Ta’line iga} - The second day\\
\textit{Jo’ine iga} - The third day\\
\textit{Nvhgine iga} - The fourth day\\
\textit{Jun(v)gilosdi} - The day they wash their clothes\\
\footnote{The first way to say Friday was actually "hisgine'iga" which means "the fifth day."}\textit{Unadodagwidena} - The day before they do something all day (when you went to town)\\
\textit{Unadodagwasgv’i} - The day they do something all day.\\
